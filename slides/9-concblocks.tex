\documentclass[aspectratio=169]{beamer}
\usepackage{etex} % fixes new-dimension error
\usepackage{lmodern}
\usepackage[T1]{fontenc}

\usepackage{graphicx,amsmath}
\usepackage{stmaryrd} % cf. interleave
\usepackage{booktabs}
\usepackage{amscd}
\usepackage{multicol}
\usepackage[absolute,overlay]{textpos}
\usepackage{alltt}
\usepackage{proof}
\usepackage{subcaption}
\usepackage{tikz}
\usepackage{tikz-cd}
\usepackage[new]{old-arrows}
\usepackage[all]{xy}
\usepackage{pgfplots}
\usepackage{textcomp}
\usepackage{cancel}

\input{src/macros/macros}
%-------------- template --------------------------------------------------
\input{src/macros/beamerconf}
%----------------------------------------------------------------------------

\begin{document}

\setLecture{4}{Basic building blocks of concurrency}


%%%%%%%%%%%%%%%%%%%%%
\section{Overview} %%
%%%%%%%%%%%%%%%%%%%%%

\begin{frame}{We are here}

  \vspace*{-2mm}

  \begin{block}{Blocks of sequential code running concurrently and sharing memory:}
    
  \begin{itemize}
    \item What is Scala?
    \item Concurrency in Java and its memory model
    \alert{\item Basic concurrency blocks and libraries}
    % \item Futures and promises
    \item \textcolor{gray}{\emph{Futures and Promises}}
    \item \textcolor{gray}{\emph{Data-Parallel Collections}}
    \item \textcolor{gray}{\emph{Reactive Programming (Concurrently)}}
    \item \textcolor{gray}{\emph{Software Transactional Memory}}
    \item Actor model
  \end{itemize}
  \end{block}
\end{frame}



% \fromBookW[scale=0.7]{32}{98mm}{39mm}
% \fromBook[scale=0.65]{32}{43mm}{98mm}{43mm}{39mm}

% \begin{frame}[fragile]\frametitle{Current thread}
% ~\\[-8mm]
% \begin{columns}
% \begin{column}{0.49\textwidth}
% \begin{lstlisting}
% ...
% \end{lstlisting}
% \end{column}
% \begin{column}{0.49\textwidth}
% ...
% \end{column}
% \end{columns}
% \end{frame}


\begin{frame}[t]\frametitle{What we will see}

  \begin{itemize}
    \item Thread pools: Executor and ExecutionContext
    \item Non-blocking synchronisation -- compare-and-set (CAS)
    \item Lazy (concurrent) values
    \item Concurrent collections
    \item Running OS processes
    \item \textcolor{gray}{\emph{Futures and Promises}}

  \end{itemize}


\end{frame}



%%%%%%%%%%%%%%%%%%%%%%%%%%%%%%%%%%%%%%%%%%%
\section{Existing thread pools in Scala} %%
%%%%%%%%%%%%%%%%%%%%%%%%%%%%%%%%%%%%%%%%%%%


\begin{frame}[fragile]\frametitle{Executor interface}
~\\[-8mm]
\begin{columns}
\begin{column}{0.54\textwidth}
\begin{lstlisting}[emph={executor},mathescape]
 Executor executor = $\textit{anExecutor}$;
 executor.execute(new RunnableTask1());
 executor.execute(new RunnableTask2());
 ...
\end{lstlisting}
\begin{lstlisting}[emph={executor}]
import scala.concurrent._
import java.util.concurrent.ForkJoinPool

object ExecutorsCreate extends App {
  val executor = new ForkJoinPool
  executor.execute(new Runnable {
    def run() = log("This task is run asynchronously.")
  })
  Thread.sleep(500) // not needed with fork:=false in SBT
}
\end{lstlisting}
\end{column}
\begin{column}{0.49\textwidth}
\begin{itemize}
  \item \alert{Executor:} can start a new thread, an existing one, or the current one
  \item Abstracts from the management of threads
  \item \structure{ExecutorService:} API that extends Executor with \code{shutdown}
    \begin{itemize}
      \item \code{executor.shutdown} $\to$ executes all tasks and then stops working threads
      \item \code{executor.awaitTermination(...)} $\to$ force termination if, after a given time, the tasks are not completed
    \end{itemize}
\end{itemize}
\end{column}
\end{columns}
\end{frame}


\begin{frame}[fragile]\frametitle{Scala's \texttt{ExecutionContext}}
~\\[-6mm]
\begin{columns}
\begin{column}{0.74\textwidth}
\begin{lstlisting}[emph={execute,sleep,log,global}]
import scala.concurrent._
object ExecutionContextGlobal extends App {
  val ectx = ExecutionContext.global
  ectx.execute(new Runnable {
    def run() = log("Running on the execution context.")
  })
  Thread.sleep(500)
}
\end{lstlisting}
\begin{lstlisting}[emph={execute,sleep,log,global,fromExecutorService}]
object ExecutionContextCreate extends App {
  val pool = new forkjoin.ForkJoinPool(2)
  val ectx = ExecutionContext.fromExecutorService(pool)
  ectx.execute(new Runnable {
    def run() = log("Running on the execution context again.")
  })
  Thread.sleep(500)
}
\end{lstlisting}
\end{column}
\begin{column}{0.36\textwidth}
\begin{itemize}
  \item \code{scala.concurrent}: has \alert{ExecutionContext}
  \item Similar to \structure{Executor} but more Scala specific
  \item often used as implicit parameter
  \item \alert{\code{global}}: default execution context (internally uses a \code{ForkJoinPool})
  \item \code{fromExecutorService}: creates \structure{ExecutionContext} from \structure{ExecutorService}
\end{itemize}
\end{column}
\end{columns}
\end{frame}


\begin{frame}[fragile]\frametitle{Simplifying the execution}
~\\[-8mm]
\begin{columns}[t]
\begin{column}{0.45\textwidth}
Similar to \structure{threads}:

\begin{lstlisting}[emph={thread,run,start}]
def thread(body: =>Unit): Thread = {
  val t = new Thread {
    override def run() = body
  }
  t.start()
  t
}
\end{lstlisting}
\end{column}
\begin{column}{0.56\textwidth}
We now define \alert{\code{execute}}

\begin{lstlisting}[emph={execute,sleep,log,global}]
def execute(body: =>Unit) = ExecutionContext.global.execute(
     new Runnable { def run() = body }
)
// For example:
object ExecutionContextSleep extends App {
  for (i<- 0 until 32) execute {
    Thread.sleep(2000)
    log(s"Task \§i completed.")
  }
  Thread.sleep(10000)
}
\end{lstlisting}
\end{column}
\end{columns}
\end{frame}


\begin{frame}[fragile]\frametitle{Avoid blocking indefinitely}
~\\[-8mm]
\begin{columns}
\begin{column}{0.45\textwidth}
\begin{lstlisting}[emph={execute,sleep,log,global}]
object ExecutionContextSleep
       extends App {
  for (i<- 0 until 32) execute {
    Thread.sleep(2000)
    log(s"Task \§i completed.")
  }
  Thread.sleep(10000)
}
\end{lstlisting}
\end{column}
\begin{column}{0.57\textwidth}
\begin{itemize}
  \item \structure{Expected:} all executions terminate after 2s
  \item \alert{Result:} only some execute after 2s %threads block forever (?!)
  \pause\\[4mm]
  \item Using quad-core CPU with hyper threading
  \item \code{global} has 8 threads in the thread pool
  \pause\\[4mm]
  \item executes tasks in batches of 8
  \item after 2s, 8 tasks print "completed"
  \item after 2s more, 8 more print "completed"
  \item \structure{sleep}: all enter a \structure{timed waiting state}
  \pause\\[4mm]
  \item if T1 \structure{waits} for T10 to \structure{notify}: \alert{blocks indefinitely}
\end{itemize}
\end{column}
\end{columns}
\end{frame}



%%%%%%%%%%%%%%%%%%%%%%%%%%%%%%%%%%
\section{Lock-free programming} %%
%%%%%%%%%%%%%%%%%%%%%%%%%%%%%%%%%%

\begin{frame}[fragile]\frametitle{Avoiding \texttt{syncrhonized} with atomic variables}
~\\[-8mm]
\begin{columns}
\begin{column}{0.57\textwidth}
\begin{itemize}
  \item \alert{atomic variable}: memory location that supports \alert{complex linearizable operations}
  \item ... i.e., \structure{appears to occur atomically}
  \item \code{write} of a volatile operation:
      \\ \structure{simple} linearizable operation
  \item at least two reads and/or writes:
      \\ \structure{complex} linearizable operation
  \pause
  \item \alert{java.util.concurrent.atomic} supports some complex ones:
    \begin{itemize}
       \item AtomicBoolean
       \item AtomicInteger
       \item AtomicLong
       \item AtomicReference  
    \end{itemize}
\end{itemize}
\end{column}
\begin{column}{0.45\textwidth}
~\\
\textbf{Variation of Example 1 (\code{getUniqueId})}
\begin{lstlisting}[emph={execute,sleep,log,global,AtomicLong,incrementAndGet}]
import java.util.concurrent.atomic._

object AtomicUid extends App {
  private val uid =
    new AtomicLong(0L)

  def getUniqueId(): Long =
    uid.incrementAndGet()
  execute {
    log(s"Uid asynchronously: \§{getUniqueId()}")
  }
  log(s"Got a unique id: \§{getUniqueId()}")
}
\end{lstlisting}
\end{column}
\end{columns}
\end{frame}


\begin{frame}[fragile]\frametitle{Compare-And-Set (CAS) -- the \alert{$\heartsuit$} of complex linearizable operations}
~\\[-8mm]
\begin{columns}
\begin{column}{0.48\textwidth}
\begin{itemize}
  \item CAS can be used to implement others:
    \begin{itemize}
      \item getAndSet
      \item decrementAndGet
      \item addAndGet
    \end{itemize}
  \item available in all atomic variables
  \item including AtomicReference[T]
\end{itemize}
\end{column}
\begin{column}{0.54\textwidth}
~\\
\textbf{Long-CAS conceptually equivalent to:}
\begin{lstlisting}[emph={execute,sleep,log,Long,synchronized}]
def compareAndSet(ov: Long, nv: Long):
          Boolean = this.synchronized {
    if (this.get != ov) false else {
      this.set(nv)
      true
  } }
\end{lstlisting}
\textbf{Ref-CAS conceptually equivalent to:}
\begin{lstlisting}[emph={execute,sleep,log,T,synchronized}]
def compareAndSet(ov: T, nv: T):
          Boolean = this.synchronized {
  if (!(this.get eq ov)) false else {
    this.set(nv)
    true
} }
\end{lstlisting}
\end{column}
\end{columns}
\end{frame}


\begin{frame}[fragile]\frametitle{Using CAS}
~\\[-8mm]
\begin{columns}
\begin{column}{0.48\textwidth}
\begin{itemize}
  \item Back to Example~1 (getUniqueId)
  \item Need to keep-on-trying
  \item Looks like busy-waiting, but it is much better
  \item Here: using (cheap) recursion instead of a loop
\end{itemize}
\end{column}
\begin{column}{0.54\textwidth}
~\\
\begin{lstlisting}[emph={execute,sleep,log,compareAndSet,synchronized}]
@tailrec
def getUniqueId(): Long = {
  val oldUid = uid.get
  val newUid = oldUid + 1
  if (uid.compareAndSet(oldUid, newUid)) newUid
  else getUniqueId()
}\end{lstlisting}
\end{column}
\end{columns}
\end{frame}


\begin{frame}[fragile]\frametitle{Lock-free programming -- really?}
~\\[-8mm]
\begin{columns}
\begin{column}{0.48\textwidth}
\begin{itemize}
  \item \alert{Lock-free programs}: without locks (with \code{synchronized})
  \item Achieved using \alert{atomic variables} (and some re-trying)
  \item No locks, no deadlocks...
  \pause
  \item (almost):
    \begin{itemize}
      \item lock-free $\Rightarrow$ use atomic variables (for atomicity)
      \item use atomic variables $\cancel\Rightarrow$ lock-free
    \end{itemize}
  \pause
\end{itemize}
\end{column}
\begin{column}{0.54\textwidth}
~\\
\begin{lstlisting}[emph={execute,sleep,log,compareAndSet,mySynchronized,AtomicBoolean}]
object AtomicLock extends App {
  private val lock = new AtomicBoolean(false)
  def mySynchronized(body: =>Unit): Unit = {
    while (!lock.compareAndSet(false, true)) {}
    try body finally lock.set(false)
  }
  var count = 0
  for (i<- 0 until 10) execute { mySynchronized { count += 1 } }
  Thread.sleep(1000)
  log(s"Count is: \§count")
}
\end{lstlisting}
\end{column}
\end{columns}
\end{frame}  


\begin{frame}\frametitle{Lock-freedom definition}

\begin{alertblock}{Lock-freedom}
  Given a set of threads and an operation \alert{OP}.
  \\
  \alert{OP} is \structure{lock-free} if at least one thread always completes \alert{OP} after a finite number of steps, regardless of the speed at which different threads progress.
\end{alertblock}
\end{frame}


\begin{frame}\frametitle{One more example: Concurrent filesystem}
    \begin{itemize}
      \item \textbf{Example 1:} getUniqueId()
      \item \textbf{Example 2:} Logging Bank Transfers
      \item \textbf{Example 3:} Thread pool
      \item \textbf{Example 4:} Batman
      \item \alert{\textbf{Example 5:} Concurrent filesystem}
    \end{itemize}
\end{frame}


\begin{frame}\frametitle{Concurrent filesystem}
  \splittwo{0.35}{0.68}{
    \begin{block}{Filesystem API}
      \alert{T1} is creating \structure{F}:
      \\\alert{T2} cannot copy or delete \structure{F}
      \\[4mm]\alert{T1}\,\&\,\alert{T2} are copying \structure{F}:
      \\\alert{T3} cannot delete \structure{F}
      \\[4mm]\alert{T1} is deleting \structure{F}:
      \\\alert{T2} cannot copy nor delete \structure{F}
    \end{block}
  }{
    \fromBook[scale=0.8]{79}{40mm}{121mm}{40mm}{67mm}
  }
\end{frame}


\begin{frame}[fragile]\frametitle{Concurrent filesystem -- Scala data types}
~\\[-8mm]
\begin{columns}
\begin{column}{0.35\textwidth}
    \begin{block}{Filesystem API}
      \alert{T1} is creating \structure{F}:
      \\\alert{T2} cannot copy or delete \structure{F}
      \\[4mm]\alert{T1}\,\&\,\alert{T2} are copying \structure{F}:
      \\\alert{T3} cannot delete \structure{F}
      \\[4mm]\alert{T1} is deleting \structure{F}:
      \\\alert{T2} cannot copy nor delete \structure{F}
    \end{block}
\end{column}
\begin{column}{0.68\textwidth}
\begin{lstlisting}[emph={execute,sleep,log,compareAndSet,State,AtomicReference}]
class Entry(val isDir: Boolean) {
  val state = new AtomicReference[State](new Idle)
}

sealed trait State
class Idle                extends State
class Creating            extends State
class Copying(val n: Int) extends State
class Deleting            extends State
\end{lstlisting}
\end{column}
\end{columns}
\end{frame}


\begin{frame}[fragile]\frametitle{Deleting and Copying}
\alert{Deleting:} \textbf{\structure{\underline{prepare}}} (checks for permission) then \structure{delete} (perform delete)

\alert{Copying:} \structure{aquire} (get permission); \structure{copy} (perform action); then \structure{release} (give permission)
\end{frame}


\begin{frame}[fragile]\frametitle{Prepare for deleting}

\begin{lstlisting}[emph={execute,sleep,log,logMessage,compareAndSet,State,AtomicReference,prepareForDelete}]
@tailrec
private def prepareForDelete(entry: Entry): Boolean = {
  val  s0 = entry.state.get
  s0 match {
    case i: Idle =>
      if (entry.state.compareAndSet(s0, new Deleting)) true
      else prepareForDelete(entry)
    case c: Creating =>
      logMessage("File currently created, cannot delete."); false
    case c: Copying =>
      logMessage("File currently copied, cannot delete."); false
    case d: Deleting =>
      false
  }
}
\end{lstlisting}

\cod{logMessage}: presented later -- similar to \cod{log}, but stores the log message
\end{frame}


\begin{frame}\frametitle{Bad copy -- the ABA problem}
\alert{``A\structure{B}A'' problem}: two readings of the same value \alert{A} lead to believe that \structure{B} was never present (type of race condition)

\bigskip

Illustrated by a bad acquire-release for \cod{Copying}, using a mutable \cod{n} in:
\\~~\alert{\texttt{Copying(\structure{var} n: Int)}}
\end{frame}


\begin{frame}[fragile]\frametitle{Bad code -- acquire/release Copying}
\label{slide:badcopy}
    
\begin{lstlisting}[emph={execute,sleep,log,logMessage,compareAndSet,releaseCopy}]
def releaseCopy(e: Entry): Copying = e.state.get match {
  case c: Copying =>
    val nstate = if (c.n == 1) new Idle else new Copying(c.n - 1)
    if (e.state.compareAndSet(c, nstate)) c // returns the old Copying state
    else releaseCopy(e)
}
\end{lstlisting}
\begin{lstlisting}[emph={execute,sleep,log,logMessage,compareAndSet,acquireCopy}]
def acquireCopy(e: Entry, c: Copying) = e.state.get match {
  case i: Idle =>
    c.n = 1         // updating `n` in `c`
    if (!e.state.compareAndSet(i, c)) acquireCopy(e, c)
  case oc: Copying =>
    c.n = oc.n + 1  // updating `n` in `c`
    if (!e.state.compareAndSet(oc, c)) acquireCopy(e, c)
}
\end{lstlisting}
Optimization: reusing previous \cod{Copying} if possible
\end{frame}


% \begin{frame}\frametitle{Bad trace -- T1\&T2 release; T3\&T1 aquire -- (T2 is slow)}
\begin{frame}\frametitle{Bad trace}
% -- T1 release (T2 starts rel.); T3 acquires; T1 acquire; T2 releases wrongly}

    \begin{block}{
      \wrap{T1 releases\\(\alert{T2 starts rel.})}
      ~~~$\Rightarrow$~~~
      \wrap{T3 acquires}
      ~~~$\Rightarrow$~~~
      \wrap{T1 acquires}
      ~~~$\Rightarrow$~~~
      \wrap{\alert{T2 releases}\\\alert{(wrongly)}}}
    \fromBookW{82}{151mm}{33mm}
    \end{block}
\end{frame}


\begin{frame}\frametitle{Some guidelines to avoid the ABA problem}
  \begin{itemize}
    \item use fresh objects in \cod{AtomicReference}
    \item use immutable objects in \cod{AtomicReference}
    \item avoid re-assigning the same value to an atomic variable
    \item only increment or decrement values of numeric atomic variables (if possible)
  \end{itemize}


\end{frame}


%%%%%%%%%%%%%%%%%%%%%%%%%%%%%%%%%%%%%%%%%%%%
\section{Creating and handling processes} %%
%%%%%%%%%%%%%%%%%%%%%%%%%%%%%%%%%%%%%%%%%%%%

\begin{frame}\frametitle{Beyond JVM}

\begin{itemize}
  \item \structure{So far:} run in a single JVM
  \item \structure{Now:} run processes outside JVM
  \item \structure{Why:}
  \pause
    \begin{itemize}
      \item Some programs do not exist in Scala/Java
      \item \pause Want to sandbox untrusted code
      \item \pause Performance (running independent code)
    \end{itemize}
  \pause
  \item Using the \alert{\code{scala.sys.process}} package
\end{itemize}
\end{frame}

\begin{frame}[fragile]\frametitle{Using processes - examples}
    
\begin{lstlisting}[emph={execute,sleep,log,destroy,!}]
import scala.sys.process._
object ProcessRun extends App {
  val command = "ls"
  val exitcode = command.! // run process (with side effects)
  log(s"command exited with status $exitcode") }
\end{lstlisting}

\begin{lstlisting}[emph={execute,sleep,log,destroy,!}]
def lineCount(filename: String): Int = {
  val output = s"wc $filename".!! // run and retreive stdout
  output.trim.split(" ").head.toInt }
\end{lstlisting}

\begin{lstlisting}[emph={execute,sleep,log,destroy}]
object ProcessAsync extends App {
  val lsProcess = "ls -R /".run() // run and returns a Process object
  Thread.sleep(1000)
  log("Timeout - killing ls!")
  lsProcess.destroy() } // kill a slow process
\end{lstlisting}

{\small\url{https://www.scala-lang.org/api/2.13.x/scala/sys/process/ProcessBuilder.html}}

\end{frame}


%%%%%%%%%%%%%%%%%%%%%%%%
\section{Lazy values} %%
%%%%%%%%%%%%%%%%%%%%%%%%

\begin{frame}[fragile]\frametitle{Lazy values can cause deadlocks}
~\\[-8mm]
\begin{columns}
\begin{column}{0.48\textwidth}
\begin{itemize}
  \item \alert{\textbf{lazy values:}} initialized when read for the first time
  \item these should not depend-on/modify state (non-determinism)
  \item code in \structure{singleton objects}: lazy execution
  \item under the hood: first write uses a lock -- to ensure at most a thread initialises a lazy value
  \item \structure{stack overflow} (sequential code)\\~~can become\\
        \alert{deadlock} (concurrent code)

\end{itemize}
\end{column}
\begin{column}{0.54\textwidth}
~\\
\begin{lstlisting}[emph={execute,sleep,log,compareAndSet,mySynchronized,AtomicBoolean}]
object LazyValsCreate extends App {
  var x = 5
  lazy val y = x+2
  execute {log(s"Wrk: y = $y")}
  x = 10
  log(s"Main: y = $y")
  // y = 7 or 12 in both cases
  Thread.sleep(500)
}
\end{lstlisting}
\begin{lstlisting}[emph={execute,sleep,log,compareAndSet,mySynchronized,AtomicBoolean}]
object LazyValsDeadlock extends App {
  object A { lazy val x: Int = B.y }
  object B { lazy val y: Int = A.x }
  execute { B.y }
  A.x
}
\end{lstlisting}
\end{column}
\end{columns}
\end{frame}



%%%%%%%%%%%%%%%%%%%%%%%%%%%%%%%%%%%%%%%%%%%%%
\section{Concurrent (mutable) collections} %%
%%%%%%%%%%%%%%%%%%%%%%%%%%%%%%%%%%%%%%%%%%%%%


\begin{frame}[fragile]\frametitle{Default mutable collections are not concurrent}


~\\[-8mm]
\begin{columns}
\begin{column}{0.48\textwidth}
\begin{itemize}
  \item Naive code: arbitrarily returns different results and exceptions
  \item Corruption of the internal state
  \item Possible fixes:
    \begin{itemize}
      \item immutable collections + atomic variables
      \\\only<2>{\structure{(does not scale)}}
      \item mutable collections + synchronized
      \\\only<2>{\structure{(assuming collections do not block; may not scale)}}
      \item dedicated libraries
      \\\only<2>{\structure{(far better performance and scalability)}}
    \end{itemize}
\end{itemize}
\end{column}
\begin{column}{0.54\textwidth}
~\\
\begin{lstlisting}[emph={execute,sleep,log,compareAndSet,asyncAdd}]
import scala.collection._
object CollectionsBad extends App {
  val buffer = mutable.ArrayBuffer[Int]()
  def asyncAdd(numbers: Seq[Int]) = execute {
    buffer ++= numbers
    log(s"buffer = $buffer")
  }
  asyncAdd(0 until 10)
  asyncAdd(10 until 20)
  Thread.sleep(500)
}
\end{lstlisting}
\end{column}
\end{columns}
\end{frame}


\begin{frame}[t]\frametitle{Some concurrent collections}
    
  \begin{itemize}
    \item Concurrent queues 
      \begin{itemize}
         \item java.util.concurrent.BlockingQueue \structure{interface}
        \item ...ArrayBlockingQueue \alert{class} (bounded)
        \item ...LinkedBlockingQueue \alert{class} (unbounded)
       \end{itemize} 
    \item Concurrent Sets and Maps
      \begin{itemize}
        \item scala.collection.concurrent.Map \structure{trait}
        \item java.util.concurrent.ConcurrentHashMap \alert{class}
      \end{itemize}
  \end{itemize}
\end{frame}

\begin{frame}[t]\frametitle{BlockingQueue API}
    \centering
    \fromBook[scale=1.2]{90}{20mm}{24mm}{59mm}{172mm}
\end{frame}

\begin{frame}[fragile]\frametitle{Back to Example 5: logging in our concurrent filesystem}
    
We will compile a queue of messages when \alert{logging} messages in our file system


\begin{lstlisting}[emph={execute,sleep,log,compareAndSet,logger,messages,logMessage}]
class FileSystem(...) {
  ...
  private val messages = new LinkedBlockingQueue[String]
  val logger = new Thread {
    setDaemon(true)
    override def run() = while (true) log(messages.take())
  }
  logger.start()
  def logMessage(msg: String): Unit = messages.offer(msg)
}
\end{lstlisting}

\begin{lstlisting}[emph={execute,sleep,log,compareAndSet,logMessage}]
...
val fileSystem = new FileSystem(".") // to be defined later
fileSystem.logMessage("Testing log!")
\end{lstlisting}
\end{frame}


\begin{frame}\frametitle{Note on iterators}

\begin{itemize}
  \item concurrentQueue.iterator
  \item can produce inconsistent results
  \item while traversing and modifying, the iterator can be updated
  \item (heavier) exceptions create a copy when producing an iterator
\end{itemize}
\end{frame}


\begin{frame}[fragile]\frametitle{Example 5: files as a Map in our FileSystem}

\begin{lstlisting}[emph={execute,sleep,log,ConcurrentHashMap,files,Entry,iterateFiles}]
import scala.collection.convert.decorateAsScala._
import java.io.File
import org.apache.commons.io.FileUtils // needs "commons-io" in build.sbt

class FileSystem(val root: String) {
  val rootDir = new File(root)
  val files: concurrent.Map[String, Entry] =
    new ConcurrentHashMap().asScala
  for (f <- FileUtils.iterateFiles(rootDir, null, false).asScala)
    files.put(f.getName, new Entry(false))

  ...
}
\end{lstlisting}
\end{frame}


\begin{frame}[fragile]\frametitle{Deleting a file}

Recall the \code{prepareForDelete(entry)}

\begin{lstlisting}[emph={execute,sleep,log,deleteFile,files,prepareForDelete,deleteQuietly}]
...
def deleteFile(filename: String): Unit = {
  files.get(filename) match {
    case None =>
      logMessage(s"Path '$filename' does not exist!")
    case Some(entry) if entry.isDir =>
      logMessage(s"Path '$filename' is a directory!")
    case Some(entry) => execute {
      if (prepareForDelete(entry))
        if (FileUtils.deleteQuietly(new File(filename)))
          files.remove(filename)
    }
  }
}
\end{lstlisting}
\end{frame}


\begin{frame}\frametitle{Some complex linearizable methods of concurrent Map}
    \centering
    \fromBookW[scale=0.8]{95}{34mm}{142mm}

    \splittwo{0.48}{0.48}{
      \begin{itemize}
        \item These use ``equals'' instead of the reference (which CAS does)
        \item Avoid \code{null} as key or value (often used as special values)
      \end{itemize}
    }{
      \begin{itemize}
        \item Methods \code{+=}, \code{-=}, \code{put}, \code{update}, \code{get}, \code{apply}, \code{remove} are (non-complex) linearizable (thread-safe)
      \end{itemize}
    }
    % (Using "equals" instead of the reference, as with CAS)
\end{frame}


\section*{Wrapping up our Filesystem (Example 5)}

\begin{frame}[fragile]\frametitle{Copying in our Filesystem}
    Recall our \alert{broken} \structure{aquireCopy}/\structure{releaseCopy} methods (\alert{ABA problem}) -- slide\ref{slide:badcopy}

\begin{lstlisting}[emph={execute,sleep,log,Copying,acquire,compareAndSet}]
@tailrec
private def acquire(entry: Entry): Boolean = {
  val s0 = entry.state.get
  s0 match {
    case _: Creating | _: Deleting =>
      logMessage("File inaccessible, cannot copy."); false
    case i: Idle =>
      if (entry.state.compareAndSet(s0, new Copying(1))) true
      else acquire(entry)
    case c: Copying =>
      if (entry.state.compareAndSet(s0, new Copying(c.n+1))) true
      else acquire(entry)
  }
}
\end{lstlisting}
\end{frame}


\begin{frame}[fragile]\frametitle{Copying in our Filesystem}
    Same CAS retry-approach for releasing.

\begin{lstlisting}[emph={execute,sleep,log,Copying,release,compareAndSet}]
@tailrec
private def release(entry: Entry): Unit = {
  val s0 = entry.state.get
  s0 match {
    case c: Creating =>
      if (!entry.state.compareAndSet(s0, new Idle)) release(entry)
    case c: Copying =>
      val nstate = if (c.n == 1) new Idle else new Copying(c.n-1)
      if (!entry.state.compareAndSet(s0, nstate)) release(entry)
  }
}
\end{lstlisting}
\end{frame}


\begin{frame}[fragile]\frametitle{Copying in our Filesystem}
    Finally: wrapper for copying a file.

\begin{lstlisting}[emph={execute,sleep,log,Copying,copyFile,acquire, release}]
def copyFile(src: String, dest: String): Unit = {
  files.get(src) match {
    case Some(srcEntry) if !srcEntry.isDir => execute {
      if (acquire(srcEntry)) try {
        val destEntry = new Entry(isDir = false)
        destEntry.state.set(new Creating)
        if (files.putIfAbsent(dest, destEntry) == None) try {
          FileUtils.copyFile(new File(src), new File(dest))
        } finally release(destEntry)
      } finally release(srcEntry)
    }
} }
\end{lstlisting}
\end{frame}



%%%%%%%%%%%%%%%%%%%%%%%%%%%%%%%%%%%%%%%%%%%%%
% \section{Wrap up} %%
%%%%%%%%%%%%%%%%%%%%%%%%%%%%%%%%%%%%%%%%%%%%%

\begin{frame}\frametitle{Wrapping up ``concurrency blocks''} %{Summary of concurrency blocks covered}
  \splittwo{0.56}{0.44}{
    \begin{itemize}
      \item \texttt{\color{blue!75!black}executor.execute(...)}
      \item lock-free programming with atomic variables
      \item \texttt{\color{blue!75!black}av.compareAndSet(...)}
      \item ABA problem
      \item Lazy values \& ``lazy'' objects
      \item \texttt{\color{blue!75!black}java.util.concurrent.BlockingQueue}
      \item \texttt{\color{blue!75!black}scala.collection.concurrent.Map}
      \item \textcolor{gray}{\emph{weakly consistent iterators}}
      \item \textcolor{gray}{\emph{custom concurrent data structures}}
    \end{itemize}
  }{
    \begin{itemize}
      \item Filesystem example
      \item Processes outside JVM
    \end{itemize}
    \pause
    \begin{alertblock}{Next}
      \begin{itemize}
        \item \textcolor{gray}{\emph{Futures and Promises}}
        \item \textcolor{gray}{\emph{Data-Parallel Collections}}
        \item \textcolor{gray}{\emph{Reactive Programming (Concurrently)}}
        \item \textcolor{gray}{\emph{Software Transactional Memory}}
        \item \alert{Actors}
      \end{itemize}
    \end{alertblock}
  }
\end{frame}



%%%%%%%%%%%%%%%%%%%%%%%%%%%%%%%%%%%%%%%%%%%%%
\section{Extra: Quick guide to the Future} %%
%%%%%%%%%%%%%%%%%%%%%%%%%%%%%%%%%%%%%%%%%%%%%

\begin{frame}[t]\frametitle{Futures and Promises}

\begin{itemize}
  \item High-level asynchronous constructs
  \item Used very often by libraries that use concurrent tasks
  \\~
  \item \alert{Future}:
    read-only placeholder for the result of an ongoing computation
  % is a placeholder that eventually becomes a value to be \alert{read}
  \item \structure{Promise}:
    writable, single-assignment container that completes a Future
  % is a placeholder that eventually becomes a value to be \structure{written}
\end{itemize}

\end{frame}

\begin{frame}[fragile]\frametitle{Creating a Feature}
~\\[-8mm]
\begin{columns}
\begin{column}{0.56\textwidth}
~\\
\begin{lstlisting}[emph={execute,sleep,log,Future}]
object LazyValsCreate extends App {
  import scala.concurrent._
  import Future
  import ExecutionContext.Implicits.global
  
  val futFive = Future {
    log("Computation started.")
    val result = {
      Thread.sleep(5000)
      5
    }
    log("Computation finally finished.")
    result // : Future[Int]
  }
}
\end{lstlisting}
\end{column}
\begin{column}{0.46\textwidth}
\begin{block}{Small notes}
\begin{itemize}
  \item Using our \texttt{ExecutionContext.global}
  \item import of \texttt{Implicits.global} $\Leftrightarrow$
  \\ \texttt{implicit val ec = ExecutionContext.global}
  \pause\\~
  \\\item{\footnotesize\texttt{Future.apply[T]}}
  \\~~{\footnotesize\texttt{(body: => T)}}
  \\~~{\footnotesize\texttt{(implicit ec: ExecutionContext)}}
  \\~~{\footnotesize\texttt{: Future[T]}}
\end{itemize}
\end{block}
\end{column}
\end{columns}
\end{frame}


\begin{frame}[fragile]\frametitle{Getting the values 1: onComplete}
~\\[-8mm]
\begin{columns}
\begin{column}{0.56\textwidth}
~\\
\begin{lstlisting}[emph={execute,sleep,log,Future,Try,Success,Failure,onComplete}]
import scala.util.{Try,Success,Failure}
...
val futFive = Future { ... }

def getFive = futFive.onComplete {
  case Success(n) =>
    log(s"Got number: $n")
  case Failure(exception) =>
    log(s"Failed to get number: $exception")
}
getFive
Thread.sleep(6000)
getFive // What to expect?
\end{lstlisting}
\end{column}
\begin{column}{0.46\textwidth}
\begin{block}{A future has 2 stages}
\begin{itemize}
  \item \alert{Not completed} (ongoing computation)
  \item \structure{Completed} (finished computation)
  \begin{itemize}
    \item \texttt{Success(res)} (no errors)
    \item \texttt{Failure(err)} (got errors)
  \end{itemize}
\end{itemize}
\end{block}
\begin{alertblock}{Method 1: onComplete}\small
  \texttt{futFive.onComplete[U]}
  \\~~\texttt{(callback: Try[Int] => U)}
  \\~~\texttt{(implicit exec: ...)}
  \\~~\texttt{: Unit}
\end{alertblock}

\end{column}
\end{columns}
\end{frame}


\begin{frame}[fragile]\frametitle{Getting the values 2: Await}
~\\[-8mm]
\begin{columns}
\begin{column}{0.56\textwidth}
~\\
\begin{lstlisting}[emph={execute,sleep,log,Future,Try,Success,Failure,onComplete,Await,result,ready}]
import scala.concurrent.duration._
...
val futFive = Future { ... }

def getFive  = // : Int
  Await.result(futFive, Duration.inf) 
def doneFive = // : Future[Int]
  Await.ready (futFive, 10.seconds)

val r1 = getFive // doneFive
val r2 = getFive // doneFive
Thread.sleep(6000)
val r3 = getFive // doneFive
\end{lstlisting}
\end{column}
\begin{column}{0.46\textwidth}
\begin{block}{A future has 2 stages}
\begin{itemize}
  \item \alert{Not completed} (ongoing computation)
  \item \structure{Completed} (finished computation)
  \begin{itemize}
    \item \texttt{Success(res)} (no errors)
    \item \texttt{Failure(err)} (got errors)
  \end{itemize}
\end{itemize}
\end{block}
\begin{alertblock}{Method 2: Await.result/ready}
  \begin{itemize}
    \item Can throw \texttt{TimeoutException}
  \end{itemize}
\end{alertblock}

\end{column}
\end{columns}
\end{frame}


\begin{frame}[fragile]\frametitle{Polling a future}
~\\[-8mm]
\begin{columns}
\begin{column}{0.56\textwidth}
~\\
\begin{lstlisting}[emph={execute,sleep,log,Future,Try,Success,Failure,onComplete,Await,result,ready}]
import scala.concurrent.duration._
...
val futFive = Future { ... }

def checkFive = // : Boolean
  futFive.isCompleted
def maybeFive = // : Option[Try[Int]]
  futFive.value

val r1 = checkFive // maybeFive
val r2 = checkFive // maybeFive
Thread.sleep(6000)
val r3 = checkFive // maybeFive
\end{lstlisting}
\end{column}
\begin{column}{0.46\textwidth}
\begin{block}{A future has 2 stages}
\begin{itemize}
  \item \alert{Not completed} (ongoing computation)
  \item \structure{Completed} (finished computation)
  \begin{itemize}
    \item \texttt{Success(res)} (no errors)
    \item \texttt{Failure(err)} (got errors)
  \end{itemize}
\end{itemize}
\end{block}
\begin{alertblock}{Polling: isCompleted/value}
  % \begin{itemize}
How to check the status without blocking
  % \end{itemize}
\end{alertblock}

\end{column}
\end{columns}
\end{frame}

\begin{frame}[fragile]\frametitle{Composing futures}
~\\[-8mm]
\begin{columns}
\begin{column}{0.56\textwidth}
~\\
\begin{lstlisting}[emph={execute,sleep,log,Future,Try,Success,Failure,onComplete,Await,result,ready}]
import scala.concurrent.duration._
...
val futTwo  = Future { ... } // takes 2s
val futFive = Future { ... } // takes 5s

val r1 = futTwo.map(x => x*3)
val r2 = for (
    x <- futFive
    y <- r1
  ) yield 
    x+y
log(s"a) \§{r1.value}/\§{r2.value}")
Thread.sleep(3000)
log(s"b) \§{r1.value}/\§{r2.value}")
Thread.sleep(6000)
log(s"c) \§{r1.value}/\§{r2.value}")
\end{lstlisting}
\end{column}
\begin{column}{0.46\textwidth}
\begin{itemize}
  \item read and update futures ``in advance''
  \item \texttt{map}, \texttt{flatMap}, \texttt{foreach}, \texttt{for}-loops
  \\~\pause
  \item very common mechanism to run tasks in parallel
  \item used in many libraries
\end{itemize}
\end{column}
\end{columns}
\end{frame}


\begin{frame}[fragile]\frametitle{Promises: the other side of the Future}
~\\[-8mm]
\begin{columns}
\begin{column}{0.46\textwidth}
\begin{block}{A promise \texttt{p} can...}
\begin{itemize}
  \item \texttt{p.future}
  \item \texttt{p.success(res)}
  \item \texttt{p.failure(exception)}
  \item \texttt{p.complete(Success(res))}
\end{itemize}
\end{block}
\end{column}\begin{column}{0.56\textwidth}
~\\
\begin{lstlisting}[emph={execute,sleep,log,Future,Promise,Try,
  Success,Failure,onComplete,Await,result,ready,success,failure},
  emph={[2]prom,fut}]
import scala.concurrent.{Future,Promise}
import scala.concurrent...Implicits.global

val prom = Promise[Int]()
val alice = Future { // execute/thread
  val res = {Thread.sleep(2000); 2}
  prom.success(res)
  log(s"gave '\§res'")
  Thread.sleep(2000) // more work
}
val fut = prom.future
val bob = Future { // execute/thread
  Thread.sleep(3000)
  fut.foreach { r =>
    log("got '\§r'")
  }
}
\end{lstlisting}
\end{column}

\end{columns}
\end{frame}

%%%%%%%%%%%%%%%%%%%%%%%%%%%%%%%%%%%%%%%%%%%%%
% \section*{Wrap up} %%
%%%%%%%%%%%%%%%%%%%%%%%%%%%%%%%%%%%%%%%%%%%%%

\begin{frame}\frametitle{Wrapping up ``concurrency blocks'' (again)}
%{Summary of concurrency blocks covered}
  \splittwo{0.56}{0.44}{
    \begin{itemize}
      \item \texttt{\color{blue!75!black}executor.execute(...)}
      \item lock-free programming with atomic variables
      \item \texttt{\color{blue!75!black}av.compareAndSet(...)}
      \item ABA problem
      \item Lazy values \& ``lazy'' objects
      \item \texttt{\color{blue!75!black}java.util.concurrent.BlockingQueue}
      \item \texttt{\color{blue!75!black}scala.collection.concurrent.Map}
      \item \textcolor{gray}{\emph{weakly consistent iterators}}
      \item \textcolor{gray}{\emph{custom concurrent data structures}}
    \end{itemize}
  }{
    \begin{itemize}
      \item Filesystem example
      \item Processes outside JVM
      \alert{\item Futures and Promises (overview)}
    \end{itemize}
    \begin{alertblock}{Next}
      \begin{itemize}
        \item \textcolor{gray}{\emph{Data-Parallel Collections}}
        \item \textcolor{gray}{\emph{Reactive Programming (Concurrently)}}
        \item \textcolor{gray}{\emph{Software Transactional Memory}}
        \item \alert{Actors}
      \end{itemize}
    \end{alertblock}
  }
\end{frame}

\end{document}

\documentclass[aspectratio=169]{beamer}

\usepackage{etex} % fixes new-dimension error
\usepackage{lmodern}
\usepackage[T1]{fontenc}

\usepackage{graphicx,amsmath}
\usepackage{stmaryrd} % cf. interleave
\usepackage{booktabs}
\usepackage{amscd}
\usepackage{multicol}
\usepackage[absolute,overlay]{textpos}
\usepackage{alltt}
\usepackage{proof}
\usepackage{subcaption}
\usepackage{tikz}
\usepackage{tikz-cd}
\usepackage[new]{old-arrows}
\usepackage[all]{xy}
\usepackage{pgfplots}
\usepackage{textcomp}

\usepackage{algorithmicx}
\usepackage[noend]{algpseudocode}
\usepackage[Algoritmo]{algorithm}

\input{src/macros/macros}

%-------------- template --------------------------------------------------
\input{src/macros/beamerconf}
%----------------------------------------------------------------------------
% \usepackage[utf8]{inputenc}
% %\usepackage[portuges]{babel}
% \usepackage{alltt}
% \usepackage{xspace}
% \usepackage{longtable}
% \usepackage{amsfonts}
% \usepackage{latexsym}
% \usepackage{amsmath}
% \usepackage{amssymb}
% %\usepackage{xcolor}
% \usepackage{enumerate}
% \usepackage{graphicx}
% %\usepackage{moreverb}
% %\usepackage{ifthen}
% \usepackage{url}
% \usepackage{tikz} 

% %\usetikzlibrary{snakes}
% \usetikzlibrary{decorations}
% \usetikzlibrary{shapes,shapes.geometric,arrows,fit,calc,positioning,automata,decorations.pathmorphing}
% \tikzset{state/.style={draw,ellipse}}

% \usepackage{algorithmicx}
% \usepackage[noend]{algpseudocode}
% \usepackage[Algoritmo]{algorithm}

\newcommand{\Versao}{pdf}

\newcommand{\lixo}[1]{}
% VFS6

\newcommand{\REs}{REs\xspace}
\newcommand{\Regfin}{\mathsf{Rfin}\xspace}
\newcommand{\dfa}{DFA\xspace}
\newcommand{\dfas}{DFAs\xspace}
\newcommand{\nfa}{NFA\xspace}
\newcommand{\enfa}{$\varepsilon$-NFA\xspace}
\newcommand{\enfas}{$\varepsilon$-NFAs\xspace}
\newcommand{\nfas}{NFAs\xspace}
\newcommand{\Set}[1]{\{\,#1\,\}}
\newcommand{\fun}{\rightarrow}
 \newcommand{\Implica}{\mbox{$\Rightarrow$}}
 \newcommand{\equival}{\mbox{$\Leftrightarrow$}}
 \newcommand{\halmos}{\mbox{$\Box$} \vspace{0.25in}}
 % \newcommand{\ie}{\mbox{\em i.e.}}
 \newcommand{\prim}{$1^{\mbox{\small \b{a}}}$\ }
 \newcommand{\eira}[1]{$#1^{\mbox{\small \b{a}}}$\ }
 \newcommand{\eiro}[1]{$#1^{\mbox{\small \b{o}}}$\ }
 \newcommand{\num}{$\mbox{n}^{\mbox{\small \b{o}}}$\ }
 % \newcommand{\set}[1]{\{#1\}}
 \newcommand{\Union}{\cup}
 \newcommand{\Inter}{\cap}
 \newcommand{\AND}{\;\land\;}
 \newcommand{\ANDe}{\;\land}
 \newcommand{\Or}{\;\lor\;}
 \newcommand{\Ore}{\;\lor}
 \newcommand{\Imp}{\to}
 \newcommand{\Impv}{\;\to\;}
 \newcommand{\Equiv}{\leftrightarrow}
 \newcommand{\True}{\mathsf{true}}
 \newcommand{\False}{\mathsf{false}}
 \newcommand{\NAND}{\tilde{\AND}}
 \newcommand{\NOR}{\tilde{\Or}}
 \newcommand{\ORE}{\dot{\lor}}
 \newcommand{\bool}{{\cal BOOL}}
 \newcommand{\EEquiv}{\Longleftrightarrow}
 \newcommand{\V}{{\color{red} \textbf{V}}}
 \newcommand{\Fa}{{\color{red} \textbf{F}}}

\newcommand{\Ident}{\equiv}



%MC
\newcommand{\kleene}[1]{\mbox{${#1}^\star$}}
\newcommand{\nter}[1] {\mbox{$\langle #1 \rangle$}}
\newcommand{\deriva}{\Rightarrow}
\newcommand{\produ}{\to}
\newcommand{\blank}{\bullet}
\newcommand{\esqa}{\mbox{$\gets$}}
\newcommand{\dira}{\mbox{$\to$}}

%LC
\newcommand{\Varp}{\mbox{${\cal V}_{Prop}$}}
\newcommand{\suf}{\mbox{$\Rightarrow$}}
\newcommand{\nec}{\mbox{$\Leftarrow$}}
\newcommand{\esimo}[2]{$#1^{\mbox{\small \b{#2}}}$\ }


%SD
\newcommand{\la}{\lambda}
\newcommand{\lc}{$\lambda$-calculus\xspace}
\newcommand{\lt}{$\lambda$-termo\xspace}
\newcommand{\lts}{$\lambda$-termos\xspace}
\newcommand{\LC}{\Lambda(\CC)}
\newcommand{\CLC}{{\cal CL}(\CC)}
\newcommand{\CL}{{\cal CL}}
\newcommand{\lapp}{\backslash}
\newcommand{\Rbu}[2]{ #1\rightarrow_{1\beta} #2 } 
\newcommand{\Rb}[2]{ #1\rightarrow_{\beta} #2 } 
\newcommand{\Ra}[2]{ #1\rightarrow_{\alpha} #2 } 
\newcommand{\Cb}[2]{ #1\equiv_{\beta} #2 } 
\newcommand{\Reu}[2]{ #1\rightarrow_{1\eta} #2 } 
\newcommand{\Ret}[2]{ #1\rightarrow_{\eta} #2 } 
\newcommand{\Ce}[2]{ #1\equiv_{\eta} #2 } 
\newcommand{\Rbeu}[2]{ #1\rightarrow_{1\beta\eta} #2 } 
\newcommand{\Rbet}[2]{ #1\rightarrow_{\beta\eta} #2 } 
\newcommand{\Cbe}[2]{ #1\equiv_{\beta\eta} #2 } 

%FLP

\newcommand{\Rw}{ \mapsto_{w} } 
\newcommand{\Rws}{ \mapsto_{w}^\star } 


\newcommand{\nums}[1]{\lceil\ #1\rceil}
\newcommand{\cpo}{($D$,$\sqsubseteq$)\ }

\newcommand{\cp}[1]{(#1,$\sqsubseteq$)}
\newcommand{\nn}{\mbox{$\mathbb{N}$}}
\newcommand{\nb}{\mbox{$\,\mathbb{N}_\bot\,$}}
\newcommand{\sse}{\Longleftrightarrow}
\newcommand{\ordu}{\sqsubseteq}
\newcommand{\nttr}{\;\nrightarrow\;}
\newcommand{\ttr}{\;\rightsquigarrow\xspace}
\newcommand{\tr}{\longrightarrow\xspace}
\newcommand{\trs}{\;\longrightarrow^\star\xspace}
\newcommand{\trtau}{\;\Rightarrow\;}
\newcommand{\tc}{\;\hookrightarrow\;}
\newcommand{\trd}{\;\rightarrow^d_v\;}
\newcommand{\confd}[2]{\langle #1,#2 \rangle}
\newcommand{\son}[3]{\confd{#1}{#2}\;\tr\;#3}
\newcommand{\sos}[4]{\confd{#1}{#2}\;\tr\;\confd{#3}{#4}}
\newcommand{\transt}[3]{#1\stackrel{#2}{\ttr}#3}
% \newcommand{\trans}[3]{#1\stackrel{#2}{\tr}#3}
\newcommand{\transs}[3]{#1\stackrel{#2}{\trs}#3}
\newcommand{\transc}[3]{#1\stackrel{#2}{\tc}#3}
\newcommand{\transi}[4]{#1\stackrel{#2}{\tr_{#4}}#3}
\newcommand{\transci}[4]{#1\stackrel{#2}{\tc_{#4}}#3}
\newcommand{\transtau}[3]{#1\stackrel{#2}{\trtau}#3}

\newcommand{\sta}[1]{\langle #1 \rangle}
\newcommand{\einf}{\stackrel{\infty}{\exists}}
\newcommand{\ainf}{\stackrel{\infty}{\forall}}
\newcommand{\sap}{(2^{AP})^\omega}
\newcommand{\sas}{(2^{AP})^\star}
\newcommand{\sam}{(2^{AP})^\plus}
\newcommand{\pw}{{\bf P}$\omega$}
\newcommand{\nat} {{\bf N}}
\newcommand{\fnn} {[\nat \longrightarrow \nat]_\bot}
\newcommand{\fff} {\fnn \longrightarrow \fnn}


\newcommand{\Lub}{\bigsqcup}


\newcommand{\rb}{\rightarrow_\beta}
\newcommand{\cb}{=_\beta}
\newcommand{\cbe}{=_{\beta\eta}}
\newcommand{\re}{\rightarrow_\eta}

%\newcommand{\vec}[1]{\overline{#1}}
\newcommand{\lsbra}{[\![}
\newcommand{\rsbra}{]\!]}
\newcommand{\bras}[1]{\lsbra  #1 \rsbra}
\newcommand{\brase}[1]{\lsbra  #1 \rsbra_\Gamma}
%\newcommand{\brass}[1]{Sat(#1)}
\newcommand{\sv}[2]{{\cal #1}\lsbra #2\rsbra}
\newcommand{\svs}[3]{{\cal #1}\lsbra #2\rsbra #3}
\newcommand{\svsc}[3]{{#1}\lsbra #2\rsbra #3}
\newcommand{\svsd}[4]{{#1}\lsbra #2\rsbra #3\, #4}
\newcommand{\svd}[2]{{#1}\lsbra #2\rsbra}
\newcommand{\svr}[3]{#1\lsbra #2\rsbra #3}
\newcommand{\svl}[2]{{\cal #1}\lsbra #2\rsbra\rho}
\newcommand{\svc}[2]{{\cal #1}\lsbra #2\rsbra\varphi\rho}
%\newcommand{\svc}[2]{{\cal #1}\lsbra #2\rsbra\rho\kappa}
%\newcommand{\svt}[2]{{\cal #1}\lsbra #2\rsbra\rho\theta}

\newcommand{\IFF}[3]{#1\,\fun \,#2,#3} 


\newcommand{\namesn}[1]{{#1}_{sn}}
\newcommand{\namess}[1]{{#1}_{ss}}
\newcommand{\namesd}[1]{{#1}_{sd}}

\newcommand{\fixp}[1] {\textsf{FIX}\; #1}

%VFS


\newenvironment{inducao}[2]{
  \begin{description}
      \setlength{\itemsep}{0cm}\setlength{\parsep}{0cm}
    \item[{\em Base.}] #1
    \item[{\em Indução.}] #2}
    {\end{description}}

\newenvironment{inducaoi}[1]{
    \begin{description}
      \setlength{\itemsep}{0cm}\setlength{\parsep}{0cm}
    \item[{\em Indução.}] #1}
    {\end{description}}

\newenvironment{inducaob}[1]{
    \begin{description}
      \setlength{\itemsep}{0cm}\setlength{\parsep}{0cm}
    \item[{\em Base.}] #1}
    {\end{description}}


\newenvironment{pequeno}{

\small

}{}

\newenvironment{mini}{

\tiny

}{}


 \newenvironment{twocol}[4]{

  \begin{tabular}{l|l}
    \begin{minipage}[t]{#1cm}
#2
    \end{minipage}
&
    \begin{minipage}[t]{#3cm}
#4
\end{minipage}
\end{tabular}
}{}






\newcounter{cont}
\newcounter{conti}

\newenvironment{dem}{{\bf Dem:} }

\newenvironment{proofu}{\setlength{\topsep}{-0.3cm}\setlength{\partopsep}{-0.3cm}\begin{description}\item[Dem.] }{\hfill $\Box$ \bigskip
\end{description}}

 \newenvironment{resolv}[1]{
 {\bf Resolução #1}

 }{}

 \newenvironment{coment}[1]{
 {\bf Comentário #1}

 }{}


\newenvironment{exerc}{\begin{exe} \it }{ 
$\diamond$ \end{exe}}

\newenvironment{exercicio}{\begin{Ex} \em }{ 
 \end{Ex}}

\newenvironment{resposta}[1]{{\parbox[t]{14cm}{{\bf R:}
  \vspace*{#1}}}}

\newenvironment{resposta1}[1]{{\parbox[t]{8cm}{{\bf R:} \vspace*{#1}}}}

\newcounter{taben}

\newenvironment{tabenum}{
\begin{enumerate}
  \setcounter{enumi}{\value{taben}}
  \renewcommand{\labelenumi}{\arabic{enumi}.}
  \renewcommand{\theenumi}{\arabic{enumi}.} }{\setcounter{taben}{\value{enumi}}\end{enumerate}}

%\usepackage{aula}

\newcounter{aula}

\newtheorem{exe}{Exercício}[aula]
\newtheorem{exem}{Exemplo}[aula]
\newtheorem{Ex}{Ex}[aula]
 \newtheorem{prob}{Problema}[aula]
 \newtheorem{lema}{Lema}[aula]
 \newtheorem{resol}{Resolução}[aula]
\newtheorem{teor}{Teorema}[aula]
\newtheorem{Def}{Definição}[aula]

\newtheorem{propos}{Proposição}[aula]
\newtheorem{cor}{Corolário}[aula]

%\newenvironment{saida}[1]{\fbox{\begin{verbatim} #1}{\end{verbatim}}}


\newcommand{\apply}{\textsc{apply}}
\newcommand{\restrict}{\textsc{restrict}}
\newcommand{\reduce}{\textsc{reduce}}
\newcommand{\foral}{\textsc{forall}}
\newcommand{\exist}{\textsc{exists}}


%\setlength{\itemsep}{0cm}

\tikzstyle{st}=[state, rectangle, rounded corners=7pt,  inner sep=4pt, minimum size=14pt]











\newboolean{portuguese}
\setboolean{portuguese}{true}
%\setboolean{portuguese}{false}

\newboolean{normal}
\setboolean{normal}{true}

\newcommand{\pten}[2]{\ifthenelse{\boolean{portuguese}}{#1}{#2}}
\pten{\usepackage[portuges]{babel}}{\usepackage[british]{babel}}


\usepackage{bussproofs}

\begin{document}

% \setLectureBasic{Cyber-Physical Computation}
\setLecture[\\\emph{(slides mainly from Nelma Moreira)}]%
    {5}{Bissimulação fraca e Congruência Observável}


\only<presentation>{
\begin{frame}{Bissimilaridade}
	Seja $TS=(S,\tr,s_0)$. Uma relação $R\subseteq S\times S$ é uma \alert{bissimulação} se  sempre que $(s,t)\in R$  (ou $s\;R\;t$) e $\alpha\in Act$:
\begin{itemize}
\item se 	$\trans{s}{\alpha}{s'}$ então existe $t'$ tal que   $\trans{t}{\alpha}{t'}$ e $(s',t')\in R$
\item  se $\trans{t}{\alpha}{t'}$ então existe $s'$ tal que   $\trans{s}{\alpha}{s'}$ e $(s',t')\in R$.
\end{itemize}

Dois estados $s$ e $t$ são \alert{bissimilares} $s\sim t$,   {\color{blue}se existe} uma bissimulação $R$ tal que $(s,t)\in R$.

Isto é $$\sim =\bigcup_{\text{R é bissimulação}}R$$
A relação $\sim$ chama-se \alert{bissimilaridade}
	
\end{frame}

\begin{frame}
	
\begin{itemize}
	\item relação de equivalência
\item relação de congruência
\item mesmo conjunto de traços
\item seja sensível a deadlocks
\end{itemize}
\begin{eqnarray*}
		P+Q&\sim& Q+P\\
	P|Q&\sim &Q|P\\
P+0&\sim& P\\
P|0&\sim& P\\
(P+Q)+R&\sim& P+(Q+R)\\
(P|Q)|R&\sim& P|(Q|R)\\
\alpha.(P+Q)&\not\sim &\alpha.P+\alpha.Q
	\end{eqnarray*}
\end{frame}
\begin{frame}{Mas como calcular a relação $\sim$}
Dado $TS=(S,\tr,s_0)$, 
como o maior ponto fixo de uma função ($FR$) monótona definida num reticulado completo ($2^{S\times S},\subseteq)$), i.e., $FR(B)=B$. Sem prova: 
	sendo $R\subseteq S\times S$
	\begin{align*}
		FR(R)&=\{ (s,t)\mid \trans{s}{\alpha}{s'} \implies \exists t'   \ \trans{t}{\alpha}{t'} \land (s',t')\in R \\ &\land \trans{t}{\alpha}{t'} \implies \exists s' \trans{s}{\alpha}{s'} \land (s',t')\in R \}
	\end{align*}
	
$$FR(S\times S)\supseteq 	FR^2(S\times S)\supseteq \cdots $$

ou ver algoritmo em Pseuco Book
\end{frame}
}

\begin{frame}{$2-Buffer$}

\only<article>{
\begin{exerc} Consideremos os seguintes processos que correspondem a "buffers" (de capacidade $1$ e $2$).}
\only<1->{\begin{eqnarray*}
	Buffer &:=& put?. get!.Buffer
\end{eqnarray*}}
\only<2-3>{\begin{eqnarray*}
	Buffer0 &:=& put?.Buffer1 \\
	Buffer1 &:=& put?.Buffer2 + get!.Buffer0\\
	Buffer2 &:=& get?.Buffer1
\end{eqnarray*}
Será que $Buffer0 \sim (Buffer|Buffer)$?}
\only<article>{\end{exerc}
}
\only<3>{
\begin{tabular}{lccr}
$\bras{Buffer0}_\Gamma$&$\bras{Buffer|Buffffer}_\Gamma$
\includegraphics[width=4cm]{src/img/nam/buffer2}&	\includegraphics[width=4cm]{src/img/nam/bufbuf}
\end{tabular}
}
\only<article>{Pode-se mostrar que sim. Calcula uma bissinmulação que contenha $(Buffer0,Buffer|Buffffer)$.}
\only<article>{\begin{exerc} Seja também}
\only<4->{
\begin{eqnarray*}
	BufferL &:=& put?. pass!.BufferL\\
	BufferR &:=& pass?. get!.BufferR
\end{eqnarray*}
Será que 
$(BufferL| BufferR)\backslash\{pass!,pass?\}\sim (Buffer|Buffer)$?
}

\only<5>{
%\begin{tabular}[{lccr}
\includegraphics[width=3cm]{src/img/nam/bufpass}	\includegraphics[width=4cm]{src/img/nam/bufbuf}
%\end{tabular}
}

\only<article>{Neste caso  pode-se mostrar que não é verdade.\end{exerc}}
\end{frame}

\begin{frame}{Bissimulação Fraca}

\only<1>{Ignora transições por $\tau$.}
\only<2->{
Seja $TS=(S,\tr,s_0)$. Uma relação $R\in S\times S$ é uma \alert{bissimulação fraca} se $(s,t)\in R$  (ou $s\;R\;t$) implica para todo $\alpha\in Act$:
\begin{itemize}
\item se $\trans{s}{\alpha}{s'}$ então existe $t'$ tal que   $\transtau{t}{\alpha}{t'}$ e $(s',t')\in R$
\item  se $\trans{t}{\alpha}{t'}$ então existe $s'$ tal que   $\transtau{s}{\alpha}{s'}$ e $(s',t')\in R$.
\end{itemize}
	}
\only<3>{Onde
\begin{block}{Transições fracas}
\begin{itemize}
	\item $\transtau{s}{\tau}{s'}$ sse $\exists n\geq 0: \transt{s}{\tau^n}{s'}$
\item  
$\transtau{s}{a}{s'}$ sse $\exists s'',s'''\in S: \transtau{s}{\tau}{s''}\trans{}{a}{s'''}\transtau{}{\tau}{s'}$
\end{itemize}
Nota que
\begin{itemize}
 \item $\tr\subseteq \trtau$
\item na definição de b.f. podemos usar sempre $\trtau$.
\item $\transtau{s}{\tau}{s}$ para todo o $s$
\end{itemize}
\end{block}}
\only<4>{
\begin{block}{Bissimilaridade fraca}
	Dois estados $s$ e $t$ são \alert{fracamente bissimilares}, e escreve-se $s\approx t$, se existe uma bissimulação fraca  $R$ que contêm $(s,t)$, i.e., $(s,t)\in R$.
\end{block}

}

\end{frame}
\begin{frame}{Exemplo}
\only<1->{	Mostrar que
\begin{itemize}
	\item $a.\tau.0\approx a.0\approx \tau.a.0$
\item  $\tau.(a.0+\tau.a.0)\approx a.0$
\end{itemize}
}
\only<2>{
\begin{center}
$a.\tau.0\approx a.0\approx \tau.a.0$

\includegraphics[width=7cm]{src/img/nam/a0tau}	
\end{center}

A laranja a relação $\trtau$ e a amarelo a bissimulação $R$.
}

\only<3>{
\begin{center}
$\tau.(a.0+\tau.a.0)\approx a.0$

\includegraphics[width=7cm]{src/img/nam/att}	
\end{center}
}
\end{frame}

\begin{frame}{Exemplo dos Buffers}
\begin{eqnarray*}
	Buffer &:=& put?. get?.Buffer
\\
	BufferL &:=& put?. pass!.BufferL\\
	BufferR &:=& pass?. get?.BufferR
\end{eqnarray*}

$(BufferL| BufferR)\backslash\{pass!,pass?\}\approx (Buffer|Buffer)$

\begin{tabular}{lccr}
\includegraphics[width=4cm]{src/img/nam/bufpass}&	\includegraphics[width=4cm]{src/img/nam/bufbuf}
\end{tabular}
	\end{frame}
\begin{frame}{Processos não  fracamente bissimilares}
$$(a!.b!.0)\not\approx (a!.b!.0+\tau.0)$$	
\only<1->{
\begin{tabular}[b]{lccr}
\begin{tikzpicture}[>=stealth, shorten >=2.5pt, auto, node distance=1.5cm, initial text={}]
\node[st,  initial] (S0){$a!.b!.0$};
\node[st][below of =S0] (S1){$b!.0$};
\node[st][below  of =S1] (S2){$0$};
\path[->](S0) edge  node {$a!$} (S1)
	(S1) edge  node {$b!$} (S2);
\end{tikzpicture}&&
\begin{tikzpicture}[>=stealth, shorten >=2.5pt, auto, node distance=1.5cm, initial text={}]
\node[st,  initial] (S0){$(a!.b!.0+ \tau.0)$};
\node[st][below  of =S0] (S1){$b!.0$};
\node[st][below  of =S1] (S2){$0$};
\path[->](S0) edge  node  {$a!$} (S1)
		 edge  [bend left] node  {$\tau$} (S2)
		(S1) edge  node  {$b!$} (S2);
\end{tikzpicture}
\end{tabular}
}
\only<2->{
\begin{itemize}
	\item $A$: $\trans{a!.b!.0+\tau.0}{\tau}{0}$
\item $D$: $\transtau{a!.b!.0}{\tau}{a!.b!.0}$\item $A$: $\trans{a!.b!.0}{a!}{b!.0}$
\item $D$: não pode jogar com  $a!$ porque $0$ não tem transição fraca com $a!$.
\end{itemize}
}
\end{frame}

\begin{frame}{Propriedades de $\approx$}
Temos que 
$$\approx =\bigcup_{\text{R é bissimulação fraca}}R$$

\only<1>{\begin{teor}
	A relação $\approx$ é de equivalência.
\end{teor}}
\only<2>{Notar que se $R$ é uma bissimulação fraca $R^{-1}$ também é.}

\only<3-4>{
\begin{teor}
	A relação $\approx$ é maior bissimulação fraca .
\end{teor}
\only<3>{Mostrar  que  $\approx$ é uma bissimulação fraca.}
}
\only<4->{\begin{teor}
	A bissimilaridade fraca é estritamente mais grosseira que a bissimilaridade forte, i.e  $\sim\subsetneq \approx$ .
\end{teor}
}
\only<5>{Cada bissimulação forte é uma bissimulação fraca. E vimos exemplos de processos que não bissimilares fortes mas que são bissimilares fracos.
A bissimilaridade fraca também é chamada de \emph{equivalência observável.}
}
\end{frame}

\begin{frame}{Será que $Spec\approx Uni$?} 
Seja $Spec:=pub!.Spec$ e
\begin{eqnarray*}
	CM&:=& coin?.coffee!.CM\\
CS&:=&pub!.coin!.coffee?.CS\\
Uni&:=& (CM|CS)\backslash\{coin,coffee\}
\end{eqnarray*} 
{\small\begin{tabular}[t]{lccr}
\begin{tikzpicture}[>=stealth, shorten >=2.5pt, auto, node distance=1.5cm, initial text={}]
\node[st,  initial] (S0){$Uni$};
\node[st][ below=0.4  of S0] (S1){$(CM|coin!.coffee?.CS)\backslash H $};
\node[st][below=0.4  of S1] (S2){$(coffee!.CM|coffee?.CS)\backslash H$};
\node[st][below=0.4  of S2] (S3){$(CM|CS)\backslash H$};
\path[->](S0) edge  node {$pub!$} (S1)
	(S1) edge  node {$\tau$} (S2)
	(S2) edge  node {$\tau$} (S3)
(S3) edge  [bend left=-90, yshift=180, swap] node {$pub!$} (S1)
;
\end{tikzpicture}&&
\begin{tikzpicture}[>=stealth, shorten >=2.5pt, auto, node distance=1.5cm, initial text={}]
\node[st,  initial] (S0){$Spec$};
\path[->](S0) edge [loop, right] node  {$pub!$} (S0);
\end{tikzpicture}
\end{tabular}
}
	
\end{frame}

\begin{frame}{$Spec\approx Uni$}

{\small\begin{tabular}[t]{lccr}
\begin{tikzpicture}[>=stealth, shorten >=2.5pt, auto, node distance=1.5cm, initial text={}]
\node[st,  initial] (S0){$Uni$};
\node[st][ right  of =S0] (S1){$1$};
\node[st][right of =S1] (S2){$2$};
\node[st][right  of =S2] (S3){$3$};
\path[->](S0) edge  node {$pub!$} (S1)
	(S1) edge  node {$\tau$} (S2)
	(S2) edge  node {$\tau$} (S3)
(S3) edge  [bend left=-90, yshift=180, swap] node {$pub!$} (S1)
;
\end{tikzpicture}&&
\begin{tikzpicture}[>=stealth, shorten >=2.5pt, auto, node distance=1.5cm, initial text={}]
\node[st,  initial] (S0){$Spec$};
\path[->](S0) edge [loop, right] node  {$pub!$} (S0);
\end{tikzpicture}
\end{tabular}
}

Jogo onde o defesa ganha. Configuração inicial $(Uni,Spec)$
\begin{itemize}
	\item $A$: $\trans{Spec}{pub!}{Spec}$
\item $D$: $\transtau{Uni}{pub!}{3}$ \hspace{1cm} $(3,Spec)$
\item $A$: $\trans{3}{pub!}{1}$
\item $D$: $\transtau{Spec}{pub!}{Spec}$ \hspace{1cm} $(1,Spec)$
\item $A$: $\trans{1}{\tau}{2}$
\item $D$: $\transtau{Spec}{\tau}{Spec}$ \hspace{1cm} $(2,Spec)$
\item $A$: $\trans{2}{\tau}{3}$
\item $D$: $\transtau{Spec}{\tau}{Spec}$ \hspace{1cm} $(3,Spec)$   (ciclo detectado)\end{itemize}
	
\end{frame}

\begin{frame}{Será que $Uni\approx Unibad$?}
Seja 

\begin{eqnarray*}
	CM&:=& coin?.coffee!.CM\\
	CMB&:=& coin?.coffee!.CMB+coin?.CMB\\	
CS&:=&pub!.coin!.coffee?.CS\\
Uni&:=& (CM|CS)\backslash\{coin,coffee\}\\
UniBad&:= & (CMB|CS)\backslash \{coin,cofee\}\\
\end{eqnarray*} 

\end{frame}

\begin{frame}

{\small
\begin{tikzpicture}[>=stealth, shorten >=2.5pt, auto, node distance=1.2cm, initial text={}]
\node[st,  initial] (S0){$UniBad$};
\node[st][ below  of =S0] (S1){$(CMB|coin!.coffee?.CS)\backslash H $};
\node[st][below of =S1] (S2){$(coffee!.CMB|coffee?.CS)\backslash H$};
\node[st][below  of =S2] (S3){$(CMB|CS)\backslash H$};
\node[st][right=1.1cm  of S1] (S4){$(CMB|coffee?.CS)\backslash H$};

\path[->](S0) edge  node {$pub!$} (S1)
	(S1) edge  node {$\tau$} (S2)
	(S1) edge  node {$\tau$} (S4)
	(S2) edge  node {$\tau$} (S3)
(S3) edge  [bend left=-90, yshift=180, swap] node {$pub!$} (S1)
;
\end{tikzpicture}
}

Neste caso $Uni\not\approx UniBad$, o que é bom porque $UniBad$ entra em deadlock e $Uni$ não.
\end{frame}


\begin{frame}{Jogo de bissimilaridade fraca $(Uni,UniBad)$}

{\small\begin{tabular}[t]{lccr}
\begin{tikzpicture}[>=stealth, shorten >=2.5pt, auto, node distance=1.3cm, initial text={}]
\node[st,  initial] (S0){$Uni$};
\node[st][ right  of =S0] (S1){$1$};
\node[st][right of =S1] (S2){$2$};
\node[st][right  of =S2] (S3){$3$};
\path[->](S0) edge  node {$pub!$} (S1)
	(S1) edge  node {$\tau$} (S2)
	(S2) edge  node {$\tau$} (S3)
(S3) edge  [bend left=-90, yshift=180, swap] node {$pub!$} (S1)
;
\end{tikzpicture}&
\begin{tikzpicture}[>=stealth, shorten >=2.5pt, auto, node distance=1.3cm, initial text={}]
\node[st,  initial] (S0){$UniB$};
\node[st][ right  of =S0] (S1){$1'$};
\node[st][right of =S1] (S2){$2'$};
\node[st][right  of =S2] (S3){$3'$};
\node[st][below=0.3  of S1] (S4){$4'$};
\path[->](S0) edge  node {$pub!$} (S1)
	(S1) edge  node {$\tau$} (S2)
    (S1) edge  node {$\tau$} (S4)
	(S2) edge  node {$\tau$} (S3)
	(S3) edge  [bend left=-50, yshift=180, swap] node {$pub!$} (S1)
;
\end{tikzpicture}
\end{tabular}
}
\only<2->{~\\O atacante ganha. Configuração inicial $(Uni,UniBad)$}
\splittwo{0.47}{0.47}{
\only<2->{\begin{itemize}
	\item $A$: $\trans{UniBad}{pub!}{1'}$
\item $D$: $\transtau{Uni}{pub!}{3}$ \hspace{1cm} $(3,1')$
\item $A$: $\trans{1'}{\tau}{4'}$
\item $D$: $\transtau{3}{\tau}{3}$ \hspace{1cm} $(3,4')$
\item $A$: $\trans{3}{pub!}{1}$
\item $D$: não pode jogar e perde
\end{itemize}}
}{
\only<3>{\small Mas neste caso temos que ver também o qual a estratégia do atacante quando o defesa na primeira jogada considera outras alternativas ($1$ ou $2$). Mostra que neste casos o atacante também consegue ganhar. }	
}
\end{frame}

\begin{frame}{A Bissimilaridade Fraca não é uma Congruência} 
\only<1-3>{Para todo $P,Q\in CCS$ se $P\approx Q$ e $\alpha\in Act$, $R\in CCS$ e $H\subseteq Com$,
\begin{eqnarray*}
	\alpha.P&\approx & \alpha.Q\\
P|R&\approx& Q|R\\
R|P &\approx & R|Q\\
P\backslash H &\approx &Q\backslash H
\end{eqnarray*}}
\only<2-3>{	
\alert{mas não implica que }
\begin{eqnarray*}
P+R&\approx& Q+R\\
R+P&\approx& R+Q
\end{eqnarray*}
}
\only<3->{Contraexemplo: $P=\tau.a!.0$, $Q=a!.0$ e 
$R=b!.0$. 
}
\only<4-5>{
\begin{itemize}
	\item Mostrar que $P\approx
Q$ considerando a relação 
$\mathcal{R}=\{(P,Q),(P',Q),(P'',Q')\}$ onde $P'=a!.0$ e $P''=Q'=0$ (mas em sistemas diferentes).
\end{itemize}
}
\only<5>{
\begin{tabular}{lccr}
\begin{tikzpicture}[>=stealth, shorten >=2.5pt, auto, node distance=1.5cm, initial text={}]
\node[st,  initial] (S0){$\tau.a!.0$};
\node[st][below of =S0] (S1){$P'=a!.0$};
\node[st][below  of =S1] (S2){$P''=0$};
\path[->](S0) edge  node {$\tau$} (S1)
	(S1) edge  node {$a!$} (S2);
\end{tikzpicture}&&
\begin{tikzpicture}[>=stealth, shorten >=2.5pt, auto, node distance=1.5cm, initial text={}]
\node[st,  initial] (S0){$a!.0$};
\node[st][below  of =S0] (S1){$Q'=0$};
\path[->](S0) edge  node  {$a!$} (S1);
\end{tikzpicture}
\end{tabular}}
\only<6->{
\begin{itemize}
	\item Mostrar que $P+R\not\approx Q+R$.
\end{itemize}
\begin{tabular}{lccr}
\begin{tikzpicture}[>=stealth, shorten >=2.5pt, auto, node distance=1.5cm, initial text={}]
\node[st,  initial] (S0){$\tau.a!.0+b!.0$};
\node[st][below=0.3 of S0] (S1){$a!.0$};
\node[st][below=0.3  of S1] (S2){$0$};
\path[->](S0) edge  node {$\tau$} (S1)
			edge [bend  left=80] node {$b!$} (S2)
	(S1) edge  node {$a!$} (S2)
	;
\end{tikzpicture}&&
\begin{tikzpicture}[>=stealth, shorten >=2.5pt, auto, node distance=1.5cm, initial text={}]
\node[st,  initial] (S0){$a!.0+b!.0$};
\node[st][below  of =S0] (S1){$0$};
\path[->](S0)  edge [bend left]  node  {$a!$} (S1)
		edge [bend right, swap]  node  {$b!$} (S1);
\end{tikzpicture}
\end{tabular}
}
\only<7>{
\begin{itemize}
\item A: 	$\trans{\tau.a!.0+b!.0}{\tau}{a!.0}$
\item D: $\transtau{a!.0+b!.0}{\tau}{a!.0+b!.0}$
\item A: $\trans{a!.0+b!.0}{b!}{0}$
\item D: não pode jogar porque não há nenhuma transição fraca de $a!.0$ por $b!$
\end{itemize}

}

\only<8>{O problema está nos processos não guardados!}
\end{frame}

\begin{frame}{Impondo a congruência do $+$}
\only<1->{\begin{block}{Congruência para a Escolha}
Dois processos $P,Q\in CCS$ são \alert{congruentes para a escolha}, e escreve-se $P\approx^+ Q$ sse para todos os $R\in CCS$ $$P+R\approx Q+R$$.
\end{block}}
\only<2->{Pode-se provar que 
\begin{itemize}
	\item $\approx^+ \subsetneq \approx$ (tomando $R=0$)
	\item $\approx^+$ é uma congruência e é a mais grosseira contida em  $\approx$
\end{itemize}
} 
\only<3>{Embora seja esta a noção de igualdade de processos que se pretende, não é fácil de usar  por causa de \emph{para todos os $R\in CCS$}. Vamos ver uma definição que é equivalente mas mais perto das outras definições de bissimulação.}
	\end{frame}
\begin{frame}{Congruência Observável}
	
\only<1->{\begin{block}{Congruência Observável}
Dois processos $P,Q\in CCS$ são congruentes para a observação, e escreve-se $P\simeq Q$ sse para todos $\alpha\in Act$ verifica-se que:
\begin{itemize}
	\item se $\trans{P}{\alpha}{P'}$ então  existe $Q'$ tal que $\transtau{Q}{\tau}{}\trans{}{\alpha}{}\transtau{}{\tau}{Q'}$ e $P'\approx Q'$.
\item se $\trans{Q}{\alpha}{Q'}$ então  existe $P'$ tal que $\transtau{P}{\tau}{}\trans{}{\alpha}{}\transtau{}{\tau}{P'}$ e $P'\approx Q'$.
\end{itemize} 
\end{block}}
\only<2->{A diferença entre $\approx$ e $\simeq$ está que os passos internos iniciais não podem ser simulados por não haver movimento (lacete de $\tau$).}
\only<3>{
\begin{teor}
Para todos os $P,Q\in CCS$ $P\simeq Q$ sse $P\approx^+ Q$.	
\end{teor}
}
\end{frame}
\begin{frame}{Exemplos}
\begin{itemize}
	\item $\tau.a.0\not\simeq a.0$
\item $P|\tau.Q\not\simeq \tau.(P|Q)$ 
\end{itemize}	
\end{frame}

\begin{frame}{Igualdade de comportamento}
\begin{itemize}[<+->]
	\item Dois processos $P,Q\in CCS$ \alert{tem o mesmo comportamento} sempre que $P\simeq Q$.
\item  É chamada a \alert{igualdade de comportamento}
\item as propriedades pretendidas:
\begin{itemize}
\item relação de equivalência
\item  relação de congruência
\item ter o mesmo conjunto de traços \alert{fracos}
\item seja sensível a deadlocks
\end{itemize}
\item $\sim \subsetneq\simeq\subseteq \approx$
\item em vez de $\simeq$ usa-se $=$.
\end{itemize}

\end{frame}
\begin{frame}{Regras algébricas de $=$ ($\tau$-Leis de Milner)}

Para além das satisfeitas por $\sim$ temos
\begin{eqnarray*}
	\alpha.\tau.P&=&\alpha.P\\
P+\tau.P&=&\tau.P\\
\alpha.(P+\tau.Q)&=&\alpha.(P+\tau.Q)+\alpha.Q 
\end{eqnarray*}	

Nota: as regras algébricas permitem provar a igualdade de processos pela aplicação das regras (em especial em $CCS_0$) ou considerá-las axiomas num sistema de dedução.
\end{frame}

\begin{frame}{Representantes Minimais}
\begin{itemize}[<+->]
	\item Para cada LTS podemos considerar o  \alert{representante minimal} o menor LTS cujo comportamento é igual ao LTS dado.
\item O representante minimal é único a menos de isomorfismo se for finito por estados (regular).
\item É obtido pelo quociente induzido pela relação de equivalência ($\simeq$).
\item Cada classe de equivalência é um estado
\end{itemize}


	
\end{frame}
\begin{frame}{Como obter o representante minimal}
Dado um processo $P$ do CCS estados-finito:
\begin{itemize}
	\item Construir o LTS atíngivel de $P$.
\item Calcular a relação $\approx$
\item Considerar cada classe de equivalência um estado: $S=\{[s]_{\simeq}\mid s\in Reach(P)\}$
\item Adicionar  transições entre duas classes sempre que uma transição exista entre os elementos da respectiva classe.
\item Podemos apagar as transições que seriam adicionadas por $\trtau$ e não estavam em $\tr$.
\item Adicionar um lacete com $\tau$ se $P$ tem uma $\tau$-transição para a sua classe $[P]_\approx$.
\end{itemize}	
\end{frame}

\begin{frame}{Exemplo $2-Buffer$}
\only<1>{\begin{eqnarray*}
	Buffer &:=& put?. get?.Buffer\\
	BufferL &:=& put?. pass!.BufferL\\
	BufferR &:=& pass?. get?.BufferR
\end{eqnarray*}}
Será que 
$(BufferL| BufferR)\backslash\{pass!,pass?\}\sim (Buffer|Buffer)$?

\begin{tabular}{lccr}
\includegraphics[width=4cm]{src/img/nam/bufpass}&	\includegraphics[width=4cm]{src/img/nam/bufbuf}\\
\end{tabular}

\only<2->{%
\splittwo{0.47}{0.47}{Sim e o LTS mínimo equivalente é
}{
\includegraphics[width=4cm]{src/img/nam/Buffer2}}
}
\end{frame}

\begin{frame}
\begin{center}
Expressividade do CCS
\end{center}
\end{frame}


\begin{frame}{Algoritmo de Peterson para a Exclusão Mútua}
\begin{columns}
\begin{column}{0.5\textwidth}
\begin{itemize}
		\item $P_1$, $P_2$ processos
		\item variáveis partilhadas $b_1, b_2$ e $k$, sendo 
		\item se $k=i$ então $P_i$ pode entrar
		\item  $P_1$  faz $k=2$ (dá o privilégio a $P_2$)
		\item e simétricamente para $P_2$
\item $b_i=true$ quando $P_i$ espera
\item  $b_i=false$ quando $P_i$ sai da zona crítica.
			\end{itemize}
\end{column}
\begin{column}{0.5\textwidth}
Processo $P_i$\\
		 \begin{algorithmic}
		\While{\bf true}
	\State {\structure{noncricital actions}}
\State{$b_i\gets \mathbf{true}$;}
\State{$k\gets j$;}
\While{$b_j \land k=j$} \State{\textbf{skip};}
\EndWhile
	\State {\color{red}critical actions}
\State$b_i\gets \mathbf{false};$
		\EndWhile		
\end{algorithmic}
\end{column}
\end{columns}
\end{frame}

\begin{frame}[shrink=.8]{Algoritmo de Peterson em CCS} 
\begin{eqnarray*}
	B_{1f}&:= & b1rf!.B_{1f}+b1wf?.B_{1f}+b1wt?.B_{1t}\\
B_{1t}&:= & b1rt!.B_{1t}+b1wf?.B_{1f}+b1wt?.B_{1t}\\
	B_{2f}&:= & b2rf!.B_{2f}+b2wf?.B_{2f}+b2wt?.B_{2t}\\
B_{2t}&:= & b2rt!.B_{2t}+b2wf?.B_{2f}+b2wt?.B_{2t}\\
	K_{1}&:= & kr1!.K_{1}+kw1?.K_{1}+ kw2?.K_{2}\\
K_{2}&:= & kr2!.K_{2}+kw1?.K_{1}+ kw2?.K_{2}\\
	P_1&:=& b1wt!.kw2!.P_{11}\\
P_{11}&:=& b2rf?.P_{12} + b2rt?.(kr2?.P_{11} +kr1?.P_{12})\\
P_{12}&:=& enter1.exit1.b1wf!.P_1\\
	P_2&:=& b2wt!.kw1!.P_{21}\\
P_{21}&:=& b1rf?.P_{22} + b1rt?.(kr1?.P_{21} +kr2?.P_{22})\\
P_{22}&:=& enter2.exit2.b2wf!.P_2
\end{eqnarray*}
	Se $k=1$ no início
\begin{eqnarray*}
Peterson&:=&(P_1|P_2|B_{1f}|B_{2f}|K_1)\backslash L	
\end{eqnarray*}
onde $L$ são todas as ações excepto as de entrada e saída.

Ficheiro no Pseco.com:
 \url{https://pseuco.com/\#/edit/remote/tkxz8fpu1t8ke56bj9uo}

 
\end{frame}

\begin{frame}{Exclusão mútua em CCS}
Se se modelar a  entrada e a saída da zona crítica fica

\begin{eqnarray*}
	MutexSpec &:=& enter1.exit1.MutexSpec + enter2.exit2.MutexSpec 
\end{eqnarray*}	

Será que $MutexSpec\approx Peterson$?

\only<2>{Não. Verifica que depois de 
$$\transtau{Peterson}{\tau}{(P_{12}|P_{21}|B_{1t}|B_{2t}|K_1)\backslash L}$$
o estado atingido não tem transição por $enter_2$.}
\only<3>{Contudo pode-se provar que são equivalentes por traços fracos.}
\end{frame}

\begin{frame}
\begin{exerc}
	Seja  $P:= acc?.del!.P$ \pten{que corresponde a um canal em que se envia e recebe mensagens}{that corresponds to a channel that sends and receives messages}.    \pten{Seja a seguinte implementação}{Consider the implementation}

\begin{tabular}[t]{lcl}
S&:=& acc?.send!.W\\
W &:= &ack?.S + error?.send!.W\\
R &:=&trans?.del!.ack!.R\\
M &:=&send?.($\tau$.E+trans!.M)\\
E &:=& error!.M\\
I &:= & (S$|$M$|$R)$\backslash$\{send,error,trans,ack\}
\end{tabular}
\begin{enumerate}
	\item 
  \pten{Desenha os diagramas dos processos}{Draw the diagrams of} $\bras{P}$ e $\bras{I}$.
  \item \pten{Indica}{Decide}, \pten{justificando}{justifying}, \pten{se}{if} $I\sim P$
\item \pten{Indica}{Say}, \pten{justificando}{justifying}, 
\pten{se}{if}
 $I\approx P$.
\end{enumerate}
\end{exerc}
\end{frame}
\begin{frame}%[allowframebreaks]
\begin{exerc}
~\\
\splittwo{0.55}{0.35}{
\pten{Codifica em}{Code in} CCS \pten{o seguinte protocolo $Sem$ para a exclusão mútua  de dois processos}{the following protocol for mutual exclusion of two processes} $P_1$ \pten{e}{and} $P_2$ \pten{usando um semáforo}{using a semaphore} $y$.\pten{ O código genérico para um processo }{The generic algorinthm for a process} $P_i$, \pten{com}{with} $i=1,2$ \pten{é}{is}:
}{
 \!\!\!$P_i$:\\[-11mm]~
\begin{algorithmic}
\While{\bf true}
  \While{$y=0$}
    \State{\textbf{skip};}
  \EndWhile
  \State{$y\gets 0$; $\mathbf{crit_i}$; $y\gets 1$;}
% 	\State {$\mathbf{crit_i}$;}
% \State{$y\gets 1$;}
\EndWhile		
\end{algorithmic}
}

\begin{enumerate}
\item 
\pten{Define processos}{Define the processes} $Y_1$ \pten{e}{and} $Y_0$ \pten{que correspondem  aos  valores}{corresponding to the values } $1$ \pten{e}{and} $0$ \pten{da variável}{of} $y$ (\pten{global}{global variable}). 
\pten{Considera  ações de leitura e de escrita}{Consider actions of read and write} \pten{(as ações podem ser de entrada ($?$) ou de saída~($!$))}{}.
\item \pten{Define os processos}{Define the processes}  $P_i$ \pten{para}{for} $i=1,2$, \pten{que apenas devem diferir no valor da ação critica}{, that only differ in the critical action}: $crit_1$ \pten{e}{and} $crit_2$ \pten{respectivamente}{respectively}.% (ou só  um $P[i]$ usando $CCS_{vp}^\mathbb{Z}$).
\item \pten{Define a execução paralela de}{Define the parallel execution of} $P_1$, $P_2$ \pten{e}{and} \pten{do processo que corresponde ao valor inicial $1$ para  $y$}{the process for $y$ when it is $1$}, \pten{e}{and}  \pten{considerando observáveis apenas as acções}{considering that the only observable actions are} $crit_1$ \pten{e}{and} $crit_2$ (\pten{todas as outras devem ser restritas}{all the other must be restricted}).
\item Verifica se o protocolo resultante é fracamente bissimilar com $MutexSpec$. % dado antes.
\end{enumerate}
\end{exerc}
\end{frame}

\end{document}
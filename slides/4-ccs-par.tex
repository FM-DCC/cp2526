\documentclass[aspectratio=169]{beamer}

\usepackage{etex} % fixes new-dimension error
\usepackage{lmodern}
\usepackage[T1]{fontenc}

\usepackage{graphicx,amsmath}
\usepackage{stmaryrd} % cf. interleave
\usepackage{booktabs}
\usepackage{amscd}
\usepackage{multicol}
\usepackage[absolute,overlay]{textpos}
\usepackage{alltt}
\usepackage{proof}
\usepackage{subcaption}
\usepackage{tikz}
\usepackage{tikz-cd}
\usepackage[new]{old-arrows}
\usepackage[all]{xy}
\usepackage{pgfplots}
\usepackage{textcomp}

\usepackage{algorithmicx}
\usepackage[noend]{algpseudocode}
\usepackage[Algoritmo]{algorithm}

\input{src/macros/macros}

%-------------- template --------------------------------------------------
\input{src/macros/beamerconf}
%----------------------------------------------------------------------------
% \usepackage[utf8]{inputenc}
% %\usepackage[portuges]{babel}
% \usepackage{alltt}
% \usepackage{xspace}
% \usepackage{longtable}
% \usepackage{amsfonts}
% \usepackage{latexsym}
% \usepackage{amsmath}
% \usepackage{amssymb}
% %\usepackage{xcolor}
% \usepackage{enumerate}
% \usepackage{graphicx}
% %\usepackage{moreverb}
% %\usepackage{ifthen}
% \usepackage{url}
% \usepackage{tikz} 

% %\usetikzlibrary{snakes}
% \usetikzlibrary{decorations}
% \usetikzlibrary{shapes,shapes.geometric,arrows,fit,calc,positioning,automata,decorations.pathmorphing}
% \tikzset{state/.style={draw,ellipse}}

% \usepackage{algorithmicx}
% \usepackage[noend]{algpseudocode}
% \usepackage[Algoritmo]{algorithm}

\newcommand{\Versao}{pdf}

\newcommand{\lixo}[1]{}
% VFS6

\newcommand{\REs}{REs\xspace}
\newcommand{\Regfin}{\mathsf{Rfin}\xspace}
\newcommand{\dfa}{DFA\xspace}
\newcommand{\dfas}{DFAs\xspace}
\newcommand{\nfa}{NFA\xspace}
\newcommand{\enfa}{$\varepsilon$-NFA\xspace}
\newcommand{\enfas}{$\varepsilon$-NFAs\xspace}
\newcommand{\nfas}{NFAs\xspace}
\newcommand{\Set}[1]{\{\,#1\,\}}
\newcommand{\fun}{\rightarrow}
 \newcommand{\Implica}{\mbox{$\Rightarrow$}}
 \newcommand{\equival}{\mbox{$\Leftrightarrow$}}
 \newcommand{\halmos}{\mbox{$\Box$} \vspace{0.25in}}
 % \newcommand{\ie}{\mbox{\em i.e.}}
 \newcommand{\prim}{$1^{\mbox{\small \b{a}}}$\ }
 \newcommand{\eira}[1]{$#1^{\mbox{\small \b{a}}}$\ }
 \newcommand{\eiro}[1]{$#1^{\mbox{\small \b{o}}}$\ }
 \newcommand{\num}{$\mbox{n}^{\mbox{\small \b{o}}}$\ }
 % \newcommand{\set}[1]{\{#1\}}
 \newcommand{\Union}{\cup}
 \newcommand{\Inter}{\cap}
 \newcommand{\AND}{\;\land\;}
 \newcommand{\ANDe}{\;\land}
 \newcommand{\Or}{\;\lor\;}
 \newcommand{\Ore}{\;\lor}
 \newcommand{\Imp}{\to}
 \newcommand{\Impv}{\;\to\;}
 \newcommand{\Equiv}{\leftrightarrow}
 \newcommand{\True}{\mathsf{true}}
 \newcommand{\False}{\mathsf{false}}
 \newcommand{\NAND}{\tilde{\AND}}
 \newcommand{\NOR}{\tilde{\Or}}
 \newcommand{\ORE}{\dot{\lor}}
 \newcommand{\bool}{{\cal BOOL}}
 \newcommand{\EEquiv}{\Longleftrightarrow}
 \newcommand{\V}{{\color{red} \textbf{V}}}
 \newcommand{\Fa}{{\color{red} \textbf{F}}}

\newcommand{\Ident}{\equiv}



%MC
\newcommand{\kleene}[1]{\mbox{${#1}^\star$}}
\newcommand{\nter}[1] {\mbox{$\langle #1 \rangle$}}
\newcommand{\deriva}{\Rightarrow}
\newcommand{\produ}{\to}
\newcommand{\blank}{\bullet}
\newcommand{\esqa}{\mbox{$\gets$}}
\newcommand{\dira}{\mbox{$\to$}}

%LC
\newcommand{\Varp}{\mbox{${\cal V}_{Prop}$}}
\newcommand{\suf}{\mbox{$\Rightarrow$}}
\newcommand{\nec}{\mbox{$\Leftarrow$}}
\newcommand{\esimo}[2]{$#1^{\mbox{\small \b{#2}}}$\ }


%SD
\newcommand{\la}{\lambda}
\newcommand{\lc}{$\lambda$-calculus\xspace}
\newcommand{\lt}{$\lambda$-termo\xspace}
\newcommand{\lts}{$\lambda$-termos\xspace}
\newcommand{\LC}{\Lambda(\CC)}
\newcommand{\CLC}{{\cal CL}(\CC)}
\newcommand{\CL}{{\cal CL}}
\newcommand{\lapp}{\backslash}
\newcommand{\Rbu}[2]{ #1\rightarrow_{1\beta} #2 } 
\newcommand{\Rb}[2]{ #1\rightarrow_{\beta} #2 } 
\newcommand{\Ra}[2]{ #1\rightarrow_{\alpha} #2 } 
\newcommand{\Cb}[2]{ #1\equiv_{\beta} #2 } 
\newcommand{\Reu}[2]{ #1\rightarrow_{1\eta} #2 } 
\newcommand{\Ret}[2]{ #1\rightarrow_{\eta} #2 } 
\newcommand{\Ce}[2]{ #1\equiv_{\eta} #2 } 
\newcommand{\Rbeu}[2]{ #1\rightarrow_{1\beta\eta} #2 } 
\newcommand{\Rbet}[2]{ #1\rightarrow_{\beta\eta} #2 } 
\newcommand{\Cbe}[2]{ #1\equiv_{\beta\eta} #2 } 

%FLP

\newcommand{\Rw}{ \mapsto_{w} } 
\newcommand{\Rws}{ \mapsto_{w}^\star } 


\newcommand{\nums}[1]{\lceil\ #1\rceil}
\newcommand{\cpo}{($D$,$\sqsubseteq$)\ }

\newcommand{\cp}[1]{(#1,$\sqsubseteq$)}
\newcommand{\nn}{\mbox{$\mathbb{N}$}}
\newcommand{\nb}{\mbox{$\,\mathbb{N}_\bot\,$}}
\newcommand{\sse}{\Longleftrightarrow}
\newcommand{\ordu}{\sqsubseteq}
\newcommand{\nttr}{\;\nrightarrow\;}
\newcommand{\ttr}{\;\rightsquigarrow\xspace}
\newcommand{\tr}{\longrightarrow\xspace}
\newcommand{\trs}{\;\longrightarrow^\star\xspace}
\newcommand{\trtau}{\;\Rightarrow\;}
\newcommand{\tc}{\;\hookrightarrow\;}
\newcommand{\trd}{\;\rightarrow^d_v\;}
\newcommand{\confd}[2]{\langle #1,#2 \rangle}
\newcommand{\son}[3]{\confd{#1}{#2}\;\tr\;#3}
\newcommand{\sos}[4]{\confd{#1}{#2}\;\tr\;\confd{#3}{#4}}
\newcommand{\transt}[3]{#1\stackrel{#2}{\ttr}#3}
% \newcommand{\trans}[3]{#1\stackrel{#2}{\tr}#3}
\newcommand{\transs}[3]{#1\stackrel{#2}{\trs}#3}
\newcommand{\transc}[3]{#1\stackrel{#2}{\tc}#3}
\newcommand{\transi}[4]{#1\stackrel{#2}{\tr_{#4}}#3}
\newcommand{\transci}[4]{#1\stackrel{#2}{\tc_{#4}}#3}
\newcommand{\transtau}[3]{#1\stackrel{#2}{\trtau}#3}

\newcommand{\sta}[1]{\langle #1 \rangle}
\newcommand{\einf}{\stackrel{\infty}{\exists}}
\newcommand{\ainf}{\stackrel{\infty}{\forall}}
\newcommand{\sap}{(2^{AP})^\omega}
\newcommand{\sas}{(2^{AP})^\star}
\newcommand{\sam}{(2^{AP})^\plus}
\newcommand{\pw}{{\bf P}$\omega$}
\newcommand{\nat} {{\bf N}}
\newcommand{\fnn} {[\nat \longrightarrow \nat]_\bot}
\newcommand{\fff} {\fnn \longrightarrow \fnn}


\newcommand{\Lub}{\bigsqcup}


\newcommand{\rb}{\rightarrow_\beta}
\newcommand{\cb}{=_\beta}
\newcommand{\cbe}{=_{\beta\eta}}
\newcommand{\re}{\rightarrow_\eta}

%\newcommand{\vec}[1]{\overline{#1}}
\newcommand{\lsbra}{[\![}
\newcommand{\rsbra}{]\!]}
\newcommand{\bras}[1]{\lsbra  #1 \rsbra}
\newcommand{\brase}[1]{\lsbra  #1 \rsbra_\Gamma}
%\newcommand{\brass}[1]{Sat(#1)}
\newcommand{\sv}[2]{{\cal #1}\lsbra #2\rsbra}
\newcommand{\svs}[3]{{\cal #1}\lsbra #2\rsbra #3}
\newcommand{\svsc}[3]{{#1}\lsbra #2\rsbra #3}
\newcommand{\svsd}[4]{{#1}\lsbra #2\rsbra #3\, #4}
\newcommand{\svd}[2]{{#1}\lsbra #2\rsbra}
\newcommand{\svr}[3]{#1\lsbra #2\rsbra #3}
\newcommand{\svl}[2]{{\cal #1}\lsbra #2\rsbra\rho}
\newcommand{\svc}[2]{{\cal #1}\lsbra #2\rsbra\varphi\rho}
%\newcommand{\svc}[2]{{\cal #1}\lsbra #2\rsbra\rho\kappa}
%\newcommand{\svt}[2]{{\cal #1}\lsbra #2\rsbra\rho\theta}

\newcommand{\IFF}[3]{#1\,\fun \,#2,#3} 


\newcommand{\namesn}[1]{{#1}_{sn}}
\newcommand{\namess}[1]{{#1}_{ss}}
\newcommand{\namesd}[1]{{#1}_{sd}}

\newcommand{\fixp}[1] {\textsf{FIX}\; #1}

%VFS


\newenvironment{inducao}[2]{
  \begin{description}
      \setlength{\itemsep}{0cm}\setlength{\parsep}{0cm}
    \item[{\em Base.}] #1
    \item[{\em Indução.}] #2}
    {\end{description}}

\newenvironment{inducaoi}[1]{
    \begin{description}
      \setlength{\itemsep}{0cm}\setlength{\parsep}{0cm}
    \item[{\em Indução.}] #1}
    {\end{description}}

\newenvironment{inducaob}[1]{
    \begin{description}
      \setlength{\itemsep}{0cm}\setlength{\parsep}{0cm}
    \item[{\em Base.}] #1}
    {\end{description}}


\newenvironment{pequeno}{

\small

}{}

\newenvironment{mini}{

\tiny

}{}


 \newenvironment{twocol}[4]{

  \begin{tabular}{l|l}
    \begin{minipage}[t]{#1cm}
#2
    \end{minipage}
&
    \begin{minipage}[t]{#3cm}
#4
\end{minipage}
\end{tabular}
}{}






\newcounter{cont}
\newcounter{conti}

\newenvironment{dem}{{\bf Dem:} }

\newenvironment{proofu}{\setlength{\topsep}{-0.3cm}\setlength{\partopsep}{-0.3cm}\begin{description}\item[Dem.] }{\hfill $\Box$ \bigskip
\end{description}}

 \newenvironment{resolv}[1]{
 {\bf Resolução #1}

 }{}

 \newenvironment{coment}[1]{
 {\bf Comentário #1}

 }{}


\newenvironment{exerc}{\begin{exe} \it }{ 
$\diamond$ \end{exe}}

\newenvironment{exercicio}{\begin{Ex} \em }{ 
 \end{Ex}}

\newenvironment{resposta}[1]{{\parbox[t]{14cm}{{\bf R:}
  \vspace*{#1}}}}

\newenvironment{resposta1}[1]{{\parbox[t]{8cm}{{\bf R:} \vspace*{#1}}}}

\newcounter{taben}

\newenvironment{tabenum}{
\begin{enumerate}
  \setcounter{enumi}{\value{taben}}
  \renewcommand{\labelenumi}{\arabic{enumi}.}
  \renewcommand{\theenumi}{\arabic{enumi}.} }{\setcounter{taben}{\value{enumi}}\end{enumerate}}

%\usepackage{aula}

\newcounter{aula}

\newtheorem{exe}{Exercício}[aula]
\newtheorem{exem}{Exemplo}[aula]
\newtheorem{Ex}{Ex}[aula]
 \newtheorem{prob}{Problema}[aula]
 \newtheorem{lema}{Lema}[aula]
 \newtheorem{resol}{Resolução}[aula]
\newtheorem{teor}{Teorema}[aula]
\newtheorem{Def}{Definição}[aula]

\newtheorem{propos}{Proposição}[aula]
\newtheorem{cor}{Corolário}[aula]

%\newenvironment{saida}[1]{\fbox{\begin{verbatim} #1}{\end{verbatim}}}


\newcommand{\apply}{\textsc{apply}}
\newcommand{\restrict}{\textsc{restrict}}
\newcommand{\reduce}{\textsc{reduce}}
\newcommand{\foral}{\textsc{forall}}
\newcommand{\exist}{\textsc{exists}}


%\setlength{\itemsep}{0cm}

\tikzstyle{st}=[state, rectangle, rounded corners=7pt,  inner sep=4pt, minimum size=14pt]











\usepackage{bussproofs}

\begin{document}

% \setLectureBasic{Cyber-Physical Computation}
\setLecture[\\\emph{(slides mainly from Nelma Moreira)}]%
    {4}{CCS: Calculus of Communicating Systems (parallel)}


\begin{frame}{Operadores do CCS}
\begin{description}
\item["$.\;$" Prefixo ] a execução de $\alpha.P$ começa com a execução da ação $\alpha\in Act$ e depois comporta-se como $P$	
\item["$+\;$" Escolha ] O processo $P+Q$ comporta-se como o processo $P$ ou o processo $Q$. É a escolha não determinística
\item[\alert{"$|\;$"  Composição Paralela} ]
O processo $P| Q$ representa a execução concorrente de $P$ e $Q$ (que progridem independentemente no tempo). 
\end{description}
\end{frame}

\only<presentation>{
\begin{frame}
\begin{exem}{Exemplos de $1-Buffer$ (de uma posição)}

Considera as  equações de $\Gamma$ seguintes.
\begin{enumerate}
\item Calcula $\bras{Buffer}_\Gamma$.
\begin{eqnarray*}
	Buffer &:=& put?. get?.Buffer
\end{eqnarray*}

\item Calcula $\bras{put?.BufferM}_\Gamma$.
\begin{eqnarray*}
	BufferM &:=& put?. get?.BufferM + get?.put?.BufferM
\end{eqnarray*}
\item Calcula $\bras{Buffer0}_\Gamma$.
\begin{eqnarray*}
	Buffer0 &:=& put?.Buffer1 \\
	Buffer1 &:=& get?.Buffer2 + get?.Buffer0\\
	Buffer2 &:=& get?.Buffer1\\
\end{eqnarray*}

\end{enumerate}
\end{exem}
\end{frame}

}
\begin{frame}\frametitle{Composição paralela}
	
Vimos que a composição paralela (intercalada) de dois sistemas de transição correspondia ao seu produto.

\splittwo{0.47}{0.47}{
\only<1-2>{
Por exemplo, tendo dois fósforos: Um acende, outro não

\begin{tikzpicture}[>=stealth, shorten >=1pt, auto, node distance=1.5cm,initial text={}]
\node[st, initial] (S0){};
\node[st][right of =S0] (S1){};
\node[st][right of =S1] (S2){};
\path[->](S0) edge  node {\color{red} strike}(S1)
	  (S1) edge   node { \color{red} $light$ }(S2);
\end{tikzpicture}


\begin{tikzpicture}[>=stealth, shorten >=1pt, auto, node 
distance=1.5cm,initial text={}]
\node[st, initial] (S0){};
\node[st][right of =S0] (S1){};
\node[st][right of =S1] (S2){};
\path[->](S0) edge  node {\color{red} strike}(S1)
	  (S1) edge   node { \color{red} $\tau$ }(S2);
\end{tikzpicture}

}
\only<3->{Podiamos então ter o seguinte termo do $CCS_0$:
\begin{align*}
	strike.light.strike.\tau.0+strike.strike.light.\tau.0 +\\
	strike.\tau.strike.light.0+strike.strike.\tau.light.0 \\
	\end{align*}}
\only<4>{Mas vamos considerar um operador próprio que permite mais flexibilidade e termos mais concisos. No caso anterior seria:
$$strike.light.0| strike.\tau.0$$
}

}{
\only<2->{ O produto seria então:

\begin{tikzpicture}[>=stealth, shorten >=2pt, auto, node distance=1.5cm, initial text={}]
\node[st,  initial] (S0){};
\node[st][ right of =S0] (S1){};
\node[st][right of =S1] (S2){};
\node[st][below  of =S0] (S3){};
\node[st][right  of =S3] (S4){};
\node[st][right  of =S4] (S5){};
\node[st][below  of =S3] (S6){};
\node[st][right  of =S6] (S7){};
\node[st][right  of =S7] (S8){};


\path[->](S0) edge node { \color{red} $strike$} (S1)
edge  node [swap] { \color{red} $strike$} (S3)

	(S1) edge  node { \color{red} $light$} (S2)
	  edge  node  { \color{red} $strike$} (S4)
		(S2) edge  node { \color{red} $strike$}(S5)
	(S3) edge  node { \color{red} $strike$}(S4)
	    edge  node { \color{red} $\tau$}(S6)
	 (S4)    edge  node { \color{red} $light$} (S5)
	  edge  node { \color{red} $\tau$}(S7)
	  (S5) edge  node { \color{red} $\tau$}(S8)
	  (S6) edge  node { \color{red} $strike$} (S7)
	  (S7) edge  node { \color{red} $light$} (S8)
	    ;
\end{tikzpicture}
}	
}

\end{frame}


\begin{frame}{Paralelismo}

\only<1->{ Mais dois fósforos:  $strike.(light.0 |\tau.0)$}

\only<2->{\begin{center}
\begin{tikzpicture}[>=stealth, shorten >=2.5pt, auto, node distance=2.5cm, initial text={}]
\node[st,  initial] (S0){$strike.(light.0 |\tau.0)$};
\node[st][ below  left of =S0] (S1){$light.0 |\tau.0$};
\node[st][right of =S1] (S2){$0|\tau.0$};
\node[st][below left  of =S1] (S3){$light.0|0$};
\node[st][right  of =S3] (S4){$0|0$};

\path[->](S0) edge [swap] node {$strike$} (S1)
	(S1) edge  node {$light$} (S2)
		edge  node {$\tau$} (S3)
	(S2) edge  node {$\tau$}(S4)
	(S3) edge  node {$light$}(S4);
\end{tikzpicture}\end{center}}
\only<3>{Supomos sempre que todas as ações são instântaneas.}
\end{frame}

\begin{frame}{Regras de inferência $Par$}
\begin{prooftree}
	\LeftLabel{ParE} \AxiomC{$\trans{P}{\alpha}{P'}$}
\UnaryInfC{$\trans{P|Q}{\alpha}{P'|Q}$}
\end{prooftree}
\begin{prooftree}
	\LeftLabel{ParD} \AxiomC{$\trans{Q}{\alpha}{Q'}$}
\UnaryInfC{$\trans{P|Q}{\alpha}{P|Q'}$}
\end{prooftree}
\only<article>{\begin{exerc}
Usando as regras $Par$ e as anteriormente dadas calcula o LTS $\bras{strike.(light.0 |\tau.0)}$. 	
\end{exerc}}

\end{frame}

\begin{frame}{Sincronia em $CCS$}

Para que dois sistemas possam comunicar temos de implementar algum tipo de sincronia, neste caso de \emph{handshaking}: dois processos comunicam com um par de ações que partilham o mesmo "nome". 
\only<2->{Seja
\begin{itemize}
	\item O conjunto de ações observáveis $Com$ é dividido em dois
\item $Com = A^! \cup A^?$
\item $A^!$ conjunto de ações de saída  (\emph{output}) (\alert{envio})
\item  $A^?$ conjunto de ações de entrada (\emph{input}) (\alert{recebidas})
\item Ações com o mesmo nome formam um par e  são complementares:
\item um processo envia $a^!$ e e outro recebe $a^?$
\item O complemento de $a\in  A^! \cup A^?$ designa-se  por $\overline{a}$: se $a\in A^!$ então $\overline{a}\in A^?$ e vice-versa.
\item Para a ação interna $\tau$, temos $\tau=\overline{\tau}$.
\item $\forall \alpha \in Act, \overline{\overline{\alpha}}=\alpha$.
\end{itemize}}
	
\end{frame}

\begin{frame}{Regra de inferência de Sincronia}
\only<1-3>{	\begin{prooftree}
	\AxiomC{$\trans{P}{a}{P'}$} \AxiomC{$\trans{Q}{\overline{a}}{Q'}$}
\BinaryInfC{$\trans{P|Q}{\cdots}{P'|Q'}$}
\end{prooftree}}
\only<2-3>{A sincronização não é observável por processos externos. \\}

\only<3>{A ação interna $\tau$ representa a sincronização para  o exterior.}
\only<4>{	\begin{prooftree}
	\AxiomC{$\trans{P}{a}{P'}$} \AxiomC{$\trans{Q}{\overline{a}}{Q'}$}
\LeftLabel{Sync}\BinaryInfC{$\trans{P|Q}{\tau}{P'|Q'}$}
\end{prooftree}


{\flushbottom\hrule
\begin{columns}
\begin{column}{0.5\textwidth}	
\begin{prooftree}
	\LeftLabel{ParE} \AxiomC{$\trans{P}{\alpha}{P'}$}
\UnaryInfC{$\trans{P|Q}{\alpha}{P'|Q}$}
\end{prooftree}
\end{column}
\begin{column}	{0.5\textwidth}	
\begin{prooftree}
	\LeftLabel{ParD} \AxiomC{$\trans{Q}{\alpha}{Q'}$}
\UnaryInfC{$\trans{P|Q}{\alpha}{P|Q'}$}
\end{prooftree}
\end{column}
\end{columns}
}}
\end{frame}

\begin{frame}{Exemplo}
$$(a!.0|a?.0)$$

	\only<2->{\begin{center}
\begin{tikzpicture}[>=stealth, shorten >=2.5pt, auto, node distance=2.5cm, initial text={}]
\node[st,  initial] (S0){$a!.0|a?.0$};
\node[st][ below  left of =S0] (S1){$0|a?.0$};
\node[st][below right of =S0] (S2){$a!.0|0$};
\node[st][below right   of =S1] (S3){$0|0$};
\path[->](S0) edge [swap] node {$a!$} (S1)
	 edge  node {$a?$} (S2)
		edge  node {$\tau$} (S3)
	(S2) edge  node {$a!$} (S3)
	(S1) edge  node {$a?$} (S3);
\end{tikzpicture}\end{center}}

\only<2>{A regra Sync não exclui a utilização de ParD e ParE.}
\end{frame}

\begin{frame}
	{Exemplo}
	Seja o seguinte modelo de uma máquina de café e um utilizador:

        \begin{align*}
          User&:= coin!.coffee?.morning!.0\\
          Machine&:=coin?.coffee!. Machine
          \end{align*}
          
\splittwo{0.47}{0.47}{
Desenha o LTS de $\bras{(User|Machine)}_\Gamma$ usando as regras de inferência. 
}{
 \only<1>{
 \centerline{\includegraphics[width=100pt]{src/img/nam/Coffeemachine1}}}
\only<article>{ Nota que são geradas muitas transições que não correspondem ao significado desejado. O operador seguinte permite obrigar a sincronizar certas ações.}
}
\end{frame}

\begin{frame}{Operador de Restrição}
\begin{itemize}%[<+->]
	\item Proibe  que  pares de ações  observáveis ($a$ e $\overline{a}$) sejam usadas individualmente.
\item Força a sincronia (a aplicação da regra Sync)
\item $P\backslash H$ onde $P$ é um processo e $H$ um conjunto de ações de comunicação que serão proibidas (por vezes em vez de $a!,a?$ apenas se escreve $a$ em $H$).
\item As ações internas $\tau$ não podem estar em $H$
\end{itemize}	
\begin{prooftree}
	\AxiomC{$\trans{P}{\alpha}{P'}$}
\AxiomC{$\alpha \notin H$}
\LeftLabel{Res} 
\BinaryInfC{$\trans{P\backslash H}{\alpha}{P'\backslash H}$}
\end{prooftree}

\end{frame}

\begin{frame}{Exemplo de Restrição}

$$	\trans{((a!.0|a!.0 )|a?.0)\backslash\{a!,a?\}}{\tau}{((a!.0|0 )|0)\backslash\{a!,a?\}}$$

$$	\trans{((a!.0|a!.0 )|a?.0)\backslash\{a!,a?\}}{\tau}{((0|a!.0 )|0)\backslash\{a!,a?\}}$$
%
{\flushbottom\hrule
{\small\begin{columns}
\begin{column}{0.25\textwidth}	
\begin{prooftree}
	\LeftLabel{ParE} \AxiomC{$\trans{P}{\alpha}{P'}$}
\UnaryInfC{$\trans{P|Q}{\alpha}{P'|Q}$}
\end{prooftree}
\end{column}
\begin{column}	{0.25\textwidth}	
\begin{prooftree}
	\LeftLabel{ParD} \AxiomC{$\trans{Q}{\alpha}{Q'}$}
\UnaryInfC{$\trans{P|Q}{\alpha}{P|Q'}$}
\end{prooftree}
\end{column}
\begin{column}	{0.25\textwidth}	
\begin{prooftree}
	\AxiomC{$\trans{P}{a}{P'}$} \AxiomC{$\trans{Q}{\overline{a}}{Q'}$}
\BinaryInfC{$\trans{P|Q}{\cdots}{P'|Q'}$}
\end{prooftree}
\end{column}
\begin{column}	{0.25\textwidth}	
	\begin{prooftree}
	\AxiomC{$\trans{P}{\alpha}{P'}$}
\AxiomC{$\alpha \notin H$}
\LeftLabel{Res} 
\BinaryInfC{$\trans{P\backslash H}{\alpha}{P'\backslash H}$}
\end{prooftree}
\end{column}

\end{columns}
}}
\end{frame}

\begin{frame}
{Exemplo}
No exemplo anterior calcula  o LTS 	$\bras{(User|Machine)\backslash \{coin,coffeee\}}_\Gamma$ e testa no Pseuco.com.

\only<1>{\centerline{\includegraphics[width=100pt]{src/img/nam/Coffeemachine2}}}
Indica o termo que etiqueta cada estado.

\end{frame}

\begin{frame}{No PseuCo.com}

Considera as seguintes equações
\begin{eqnarray*}
Match &:= &strike. MatchOnFire\\
MatchOnFire &:=& light!. MatchOnFire + extinguish!.0\\
TwoFireCracker &:= &light?. (bang!. 0 | bang!. 0)\\
\end{eqnarray*}
Constrói o LTS a expressão abaixo e testa no Pseuco.com.

$$(Match | TwoFireCracker) \backslash \{light\} $$
	
\end{frame}

\begin{frame}[shrink=.7]
\begin{tikzpicture}[>=stealth, shorten >=2.5pt, auto, node distance=1.5cm, initial text={}]
\node[st,  initial] (S0){$(M|TFC)\backslash\{\ell\}$};
\node[st][ below  of =S0] (S1){$(MOF|TFC)\backslash \{\ell\}$};
\node[st][below  of =S1] (S2){$(MOF|b!.0|b!.0)\backslash \{\ell\}$};
\node[st][left = 0.5cm  of S2] (S4){
$(0| TOC)\backslash \{\ell\}$
};
\node[state, inner sep=1pt, minimum size=5pt][ below  = of S2] (S3){$(MOF|0|b!.0 )\backslash \{\ell\}$};

\node[state, inner sep=1pt, minimum size=5pt][right  = 0.3cm of S3] (S5){$(MOF|b!.0|0 )\backslash \{\ell\}$};

\node[state, inner sep=1pt, minimum size=5pt][left  = 0.3cm of S3] (S6){$(0|b!.0|b!.0 )\backslash \{\ell\}$};
\node[state][below =of S6] (S7){...};
\node[state][below =of S3] (S8){...};
\node[state][left =of S8] (S10){...};
\node[state][below =of S5] (S9){...};
\node[state][left =of S9] (S11){...};
\path[->](S0) edge  node {$strike$} (S1)
	(S1) edge [swap] node {$\tau$} (S2)
	(S2) edge  node {$b!$} (S3)
(S2) edge  node {$b!$} (S5)
(S2) edge  node {$ext$} (S6)
(S1) edge  node {$ext$} (S4)
;
\path[->] (S6) edge node {$b!$} (S7)
(S3) edge node {$b!$} (S8)
(S5) edge node {$b!$} (S9)
(S3) edge node {$ext$} (S10)
(S5) edge node {$ext$} (S11);
\end{tikzpicture}

\end{frame}

\begin{frame}{CCS (quase) completo}
Seja $Com = A^! \cup A^? $ conjunto de ações de comunicação,  $Act=Com\cup\{\tau\}$ um conjunto de ações, e $Var$ um conjunto de nomes (variáveis). As expressões do $CCS$ são
\begin{eqnarray*}
P&::=& 0 \;\mid\; X \;\mid \;P+P\;  \mid \;\alpha.P\;\mid\; P | P \;\mid\; P\backslash H 
\end{eqnarray*}
onde $\alpha \in Act$, $X\in Var$ e $H \subseteq Com$. Supomos um conjunto $\Gamma$  de equações $X:=P$.
	
\end{frame}

\begin{frame}{Semântica do CCS }

A semântica das expressões do $CCS$ é então $$\bras{\_}: (Var \to CCS)\fun CCS \fun  LTS_{CCS}$$ tal que $$\bras{P}_\Gamma=(CCS,\tr_\Gamma,P)$$ com 
$$LTS_{CCS}=\{(CCS,T,s)\mid T\subseteq CCS\times Act \times CCS, \land s\in CCS\}$$ onde $\tr_\Gamma$ a mais pequena relação que satisfaz as regras de inferência.

{\flushbottom\hrule
{\tiny\begin{columns}
\begin{column}{0.15\textwidth}	
\begin{prooftree}
	\LeftLabel{Pref} \AxiomC{}
\UnaryInfC{$\trans{\alpha.P}{\alpha}{P}$}
\end{prooftree}
\begin{prooftree}
	\LeftLabel{ParE} \AxiomC{$\trans{P}{\alpha}{P'}$}
\UnaryInfC{$\trans{P|Q}{\alpha}{P'|Q}$}
\end{prooftree}
\end{column}
\begin{column}	{0.15\textwidth}	
\begin{prooftree}
	\LeftLabel{EscD} \AxiomC{$\trans{Q}{\alpha}{Q'}$}
\UnaryInfC{$\trans{P+Q}{\alpha}{Q'}$}
\end{prooftree}
\begin{prooftree}
	\LeftLabel{EscE} \AxiomC{$\trans{P}{\alpha}{P'}$}
\UnaryInfC{$\trans{P+Q}{\alpha}{P'}$}
\end{prooftree}
\end{column}
\begin{column}{0.25\textwidth}	
\begin{prooftree}
\LeftLabel{Sync}	\AxiomC{$\trans{P}{a}{P'}$} \AxiomC{$\trans{Q}{\overline{a}}{Q'}$}
\BinaryInfC{$\trans{P|Q}{\tau}{P'|Q'}$}
\end{prooftree}
\begin{prooftree}
	\LeftLabel{ParD} \AxiomC{$\trans{Q}{\alpha}{Q'}$}
\UnaryInfC{$\trans{P|Q}{\alpha}{P|Q'}$}
\end{prooftree}
\end{column}
\begin{column}	{0.25\textwidth}	
\begin{prooftree}
	\AxiomC{$\trans{P}{\alpha}{P'}$}
\AxiomC{$\alpha \notin H$}
\LeftLabel{Res} 
\BinaryInfC{$\trans{P\backslash H}{\alpha}{P'\backslash H}$}
\end{prooftree}
\begin{prooftree}
	\LeftLabel{Rec} \AxiomC{$\trans{P}{\alpha}{P'}$}
\AxiomC{$\Gamma(X)=P$}
\BinaryInfC{$\trans{X}{\alpha}{P'}$}
\end{prooftree}
\end{column}

\end{columns}
}}

\end{frame}

\begin{frame}{Mais regras de prioridade}
	\begin{itemize}[<+->]
	\item $P|Q|R$ é $(P|Q)|R$
\item  $\alpha.P | Q$ é $(\alpha.P) | Q$
\item $P + R | Q$ é $(P+R) | Q$
\end{itemize}
\end{frame}

\begin{frame}{Expressividade do CCS}
\begin{itemize}[<+->]
	\item $CCS_0$ só pode produzir sistemas de transição finitos acíclicos 
\item $CCS_0^\omega$ com $\Gamma$ finito só pode produzir sistemas de transição estado-finitos (a menos de isomorfismo)
\item $CCS$ pode produzir sistemas e transição infinitos e com ramificação infinita
\item Ex: $X:=a.0|X$ tem ramificação infinita (\alert{Verifica!})
\item com mais uma operação - renomeação $P[f]$ tem a potência duma Máquina de Turing.
\end{itemize}
\end{frame}
%\only<article>{
\begin{frame}{Operador de renomeação}
\begin{itemize}[<+->]
	\item Já vimos uma máquina de café em CCS
\item $CM:= coin?.coffee!.CM$
\item Mas agora se quisermos um chocolate? 
\item Seria igual apenas mudando a accão de envio.
\item $ChM:= coin?.choco!.ChM$
\item e o mesmo para outros produtos.
\item Então podemos ter
\item $VM:= coin?.item!.VM$
\item e definir
\begin{eqnarray*}
	CM&:=& VM[coffee/item]\\
	ChM&:=& VM[choc/item]\\
	\ldots	
\end{eqnarray*}
\end{itemize}	
\end{frame}

\begin{frame}{Operador de renomeação}
Seja $f:Act\to Act$ uma função de renomeação tal que \begin{eqnarray*}
	f(\tau)&:=&\tau,\\
f(\overline{a})&:=&\overline{f(a)}\;\; \forall a\in Com.
\end{eqnarray*}
	A função $f$ pode representar-se da forma $[b_1/a_1, \ldots b_n/a_n]$ se $f(a_i)=b_i$.

Então, sendo $P$ um processo $P[f]$ é também um processo em que cada ação $a$ é substituida por $f(a)$ (assim como os complementos).
\begin{eqnarray*}
(a.B+b.B)[a/b,b/a]&=& (b.B+a.B)\\
(a.0+\overline{a}.A)[a/b]&=&(a.0+\overline{a}.A)
\end{eqnarray*}
\end{frame}

\begin{frame}{Regra de inferência para $P[f]$}
	\begin{prooftree}
	\AxiomC{$\trans{P}{\alpha}{P'}$}
\LeftLabel{Rel} 
\UnaryInfC{$\trans{P[f]}{f(\alpha)}{P'[f]}$}
\end{prooftree}

\pause
\begin{exem}
	Seja $A:= a.b.B$, calcular	
$$\bras{(A|b.a.B)+(b.A)[a/b]}_\Gamma $$
\end{exem}

\end{frame}
%}

\begin{frame}{Operadores Estáticos e Dinâmicos}
\begin{itemize}[<+->]
	\item \alert{Operadores dinâmicos}: $\cdot$ e $+$
\item Desaparecem após se efectuar uma transição
\only<3>{\begin{prooftree}
	\LeftLabel{Pref} \AxiomC{}
\UnaryInfC{$\trans{\alpha.P}{\alpha}{P}$}
\end{prooftree}
\begin{prooftree}
	\LeftLabel{EscD} \AxiomC{$\trans{Q}{\alpha}{Q'}$}
\UnaryInfC{$\trans{P+Q}{\alpha}{Q'}$}
\end{prooftree}}

\item  \alert{Operadores estáticos}: $|$ e $\backslash$ (também $[f]$)
\item Não desaparecem após ser efectuada uma transição
\only<5>{
\begin{prooftree}
\LeftLabel{Sync}	\AxiomC{$\trans{P}{a}{P'}$} \AxiomC{$\trans{Q}{\overline{a}}{Q'}$}
\BinaryInfC{$\trans{P|Q}{\tau}{P'|Q'}$}
\end{prooftree}
\begin{prooftree}
	\LeftLabel{ParD} \AxiomC{$\trans{Q}{\alpha}{Q'}$}
\UnaryInfC{$\trans{P|Q}{\alpha}{P|Q'}$}
\end{prooftree}
\begin{prooftree}
	\AxiomC{$\trans{P}{\alpha}{P'}$}
\AxiomC{$\alpha \notin H$}
\LeftLabel{Res} 
\BinaryInfC{$\trans{P\backslash H}{\alpha}{P'\backslash H}$}
\end{prooftree}}

\end{itemize}
	
\end{frame}
\begin{frame}{CCS Regular}
Não é permitida recursão sobre operadores estáticos.

\begin{eqnarray*}
P&::=& 0 \;\mid\; X \;\mid \;P+P\;  \mid \;\alpha.P\;\mid\;  R\\
R&::=& 0 \;\mid\; \;R+R\;  \mid \;\alpha.R\;\mid\;  R | R \;\mid\; R\backslash H 
\end{eqnarray*}	
\pause
\begin{propos}
Se $\Gamma$ é uma função parcial cujo contradomínio só tem expressões regulares então o $Reach(\bras{P}_\Gamma)$  é   finito por estados para todo $P\in CCS$ (i.e. o LTS atingível é finito por estados).	
\end{propos}
\pause Este é o cálculo que se usa  mais na prática.
\end{frame}

\begin{frame}{Buffer Paralelos }
\begin{itemize}[<+->]
	\item 

	Calcular $\bras{Buffer | Buffer}_\Gamma$.
para \begin{eqnarray*}
	Buffer &:=& put?. get?.Buffer
\end{eqnarray*}
\item 
\includegraphics[width=4cm]{src/img/nam/putget.png}	


\begin{eqnarray*}
	BufferL &:=& put?. pass!.BufferL\\
	BufferR &:=& pass?. get?.BufferR\\
\end{eqnarray*}
Calcular $\bras{(BufferL| BufferR)\backslash\{pass!,pass?\}}$
\end{itemize}

\end{frame}


\begin{frame}{Exemplo de um Protocolo  com Ack (Pseuco)}
\only<1->{\begin{center}\includegraphics[width=4.5cm]{src/img/nam/SendRec1}\end{center}
}	
\only<2->{
{\small\begin{eqnarray*}
Sender &:= &put? . send! . Sending\\
Sending& :=& receiveAck? . Sender\\
Receiver &:= &receive? . get? . sendAck! . Receiver\\
	Medium& :=& send? . receive! . Medium\\
AckMedium &:=& sendAck? . receiveAck! . AckMedium\\
DupMedium &:= &Medium | AckMedium\\
ProtocolG &:=& (Sender \mid Receiver \mid DupMedium) \backslash \\
&&\{send, receive, sendAck, receiveAck\}
\end{eqnarray*}}}
\end{frame}

\begin{frame}{Exemplo de um Protocolo  com Ack (Pseuco, Holger H.)}
	\begin{center}
\includegraphics[width=10cm]{src/img/nam/protocolo1}
\end{center}
\end{frame}

\begin{frame}{Exemplo de um Protocolo  com erro no meio --Pseuco}
\begin{eqnarray*}
Sender &:= &put? . send! . Sending\\
Sending& :=& receiveAck? . Sender +receiveNAck?.send!.Sending\\
Receiver &:= &receive? . get? . sendAck! . Receiver +\\
&&gargled?.sendNAck!.Receiver\\
	Medium& :=& send? . (receive! . Medium +i.garbled!.Medium)\\
AckMedium &:=& sendAck? . receiveAck! . AckMedium+\\
&&sendNAck?.receivedNAck!.AckMedium\\
DupMedium &:= &Medium | AckMedium\\
Protocol &:=& (Sender \mid Receiver \mid DupMedium) \backslash \\
&&\{send, receive, sendAck, receiveAck,\\
&&receiveNAck,sendNAck,garbled\}
\end{eqnarray*}
\end{frame}
\begin{frame}{Comportamento Externo}
Para um observador externo os processos $Buffer$, $Protocol$ e $ProtocolG$ têm o mesmo comportamento (mas LTSs diferentes, claro!).


\begin{center}
\includegraphics[width=5cm]{src/img/nam/putget.png}	
\end{center}
\end{frame}

\begin{frame}{CCS em PSeuco -ver exemplos}
\begin{center}
[MISSING]
% \includegraphics[width=12cm]{src/img/nam/CCSemPseco.png}	
\end{center}

\end{frame}

\begin{frame}
\frametitle{Exclusão Mútua}	
\begin{itemize}
	\item \alert{Secção Crítica}: porção de código que só pode ser executado por um processo num dado instante
\item Supõe-se que a execução da secção crítica por um só processo termina.
\item \alert{MUTEX}  o problema consiste em ter
\begin{enumerate}
	\item  um algoritmo de entrada $acquire\_mutex()$
\item um algoritmo de saída $release\_mutex()$
\end{enumerate}
\item Enquadrando a secção crítica garantem
\begin{description}
	\item[Exclusão mútua:] que o código da zona crítica é executado no máximo por um processo em cada instante.
\item[\emph{Starvation-freedom}:] cada processo que invoca $acquire\_mutex()$ termina, permitindo assim que os processos que querem entrar na zona crítica o possam fazer.
\end{description}
\end{itemize}
\end{frame}


\begin{frame}{Exclusão mútua em CCS}
Se se modelar a  entrada e a saída da zona crítica fica (para dois processos).

\begin{eqnarray*}
	MutexSpec &:=& enter1.exit1.MutexSpec + enter2.exit2.MutexSpec 
\end{eqnarray*}	
\end{frame}



\begin{frame}{Algoritmo de Peterson para a Exclusão Mútua}
\begin{columns}
\begin{column}{0.5\textwidth}
\begin{itemize}
		\item $P_1$, $P_2$ processos
		\item variáveis partilhadas $b_1, b_2$ e $k$, sendo 
		\item se $k=i$ então $P_i$ pode entrar
		\item  $P_1$  faz $k=2$ (dá o privilégio a $P_2$)
		\item e simétricamente para $P_2$
\item $b_i=true$ quando $P_i$ espera
\item  $b_i=false$ quando $P_i$ sai da zona crítica.
			\end{itemize}
\end{column}
\begin{column}{0.5\textwidth}
Processo $P_i$\\
		 \begin{algorithmic}
		\While{\bf true}
	\State {\color{green}noncricital actions}
\State{$b_i\gets \mathbf{true}$;}
\State{$k\gets j$;}
\While{$b_j \land k=j$} \State{\textbf{skip};}
\EndWhile
	\State {\color{red}critical actions}
\State$b_i\gets \mathbf{false};$
		\EndWhile		
\end{algorithmic}
\end{column}
\end{columns}
\end{frame}
\begin{frame}{Algoritmo de Peterson em CCS}

O CCS não tem variáveis, mas estas  são processos que comunicam a outros processos que necessitam de ler ou escrever os seus valores.  Isto é 
\begin{itemize}
	\item As variáveis  são \alert{processos} cujos estados são os seus possíveis valores
\item Para $b_1$ temos estado $B_{1t}$ se o seu valor for $\bf true$ e  o estado $B_{1f}$ se o seu valor for \textbf{false}.
\item  Outros processos podem ler ou escrever o valor de variáveis comunicando com o respectivo processo o valor pretendido 
\item Para um processo ler $\bf true$ em $b_1$ sincroniza com esse processo por uma ação (canal) $b1rt$
\item Para um processo escrever $\bf false$ em $b_1$ sincroniza com esse processo por uma ação (canal) $b1wf$
\item Análogamente se define o comportamento de $k$ que pode tomar os valores $1$ e $2$.
\end{itemize}
	
\end{frame}
\begin{frame}{Processos para as variáveis}
\only<1->{\begin{eqnarray*}
	B_{1f}&:= & b1rf!.B_{1f}+b1wf?.B_{1f}+b1wt?.B_{1t}\\
B_{1t}&:= & b1rt!.B_{1t}+b1wf?.B_{1f}+b1wt?.B_{1t}\\
\end{eqnarray*}
	}
\only<2->{\begin{eqnarray*}
	B_{2f}&:= & b2rf!.B_{2f}+b2wf?.B_{2f}+b2wt?.B_{2t}\\
B_{2t}&:= & b2rt!.B_{2t}+b2wf?.B_{2f}+b2wt?.B_{2t}\\
\end{eqnarray*}
	}
\only<3->{\begin{eqnarray*}
	K_{1}&:= & kr1!.K_{1}+kw1?.K_{1}+ kw2?.K_{2}\\
K_{2}&:= & kr2!.K_{2}+kw1?.K_{1}+ kw2?.K_{2}\\
\end{eqnarray*}
	}

\end{frame}

\begin{frame}{Processos para $P_1$ e $P_2$}
\begin{itemize}
		\item apenas modelar a entrada e saída da zona crítica
		\item  fazer a iniciallização das variáveis $b_i$ e $k$
\item supomos que não podem terminar na zona crítica ou ficar lá para sempre
\item para representar o ciclo \textbf{while} para $P_1$ temos um estado $P_{11}$ tal que
\begin{itemize}
\item ler os valores de $b_2$ e $k$
\item esperar se $b_2=\mathbf{true}\land k=2$
\item mudar de estado, $P_{12}$ 
\item em $P_{12}$ entrar e sair da zona critica
\item mas como avaliar $b_j\land k=j$? 
\item avaliamos da esquerda para  a direita: se  $b_j$ é \textbf{true} então avaliamos  se $k=j$; a  segunda não é avaliada se a primeira for falsa.
\end{itemize}
\end{itemize}

\end{frame}

\begin{frame}{Processos para $P_1$ e $P_2$ em CCS}
\only<1->{
\begin{eqnarray*}
	P_1&:=& b1wt!.kw2!.P_{11}\\
P_{11}&:=& b2rf?.P_{12} + b2rt?.(kr2?.P_{11} +kr1?.P_{12})\\
P_{12}&:=& enter1.exit1.b1wf!.P_1
\end{eqnarray*}
}
\only<2->{
\begin{eqnarray*}
	P_2&:=& b2wt!.kw1!.P_{21}\\
P_{21}&:=& b1rf?.P_{22} + b1rt?.(kr1?.P_{21} +kr2?.P_{22})\\
P_{22}&:=& enter2.exit2.b2wf!.P_2
\end{eqnarray*}
}
\end{frame}

\begin{frame}
\frametitle{Algoritmo de Peterson em CCS} 

	Se $k=1$ no inicio
\begin{eqnarray*}
Peterson&:=&(P_1|P_2|B_{1f}|B_{2f}|K_1)\backslash L	
\end{eqnarray*}
onde $L$ são todas as ações excepto as de entrada e saída na zona critica.

Ficheiro no Pseuco.com:
 \url{https://pseuco.com/\#/edit/remote/tkxz8fpu1t8ke56bj9uo}

 Será que $MutexSpec$ e $Peterson$ são equivalentes? Mas para que equivalência? Interessa-nos o comportamento observável...
\end{frame}

\begin{frame}
	\frametitle{Teste da exclusão mútua}
	Contudo podemos mostrar que no algoritmo de Peterson um processo não entra na zona critica se o outro lá está. Para tal usamos um monitor, i.e., outro processo que executando em paralelo irá detectar se há um erro:
	\begin{align*}
		MutualTest &:= enter1!.MutualTest1 + enter2!.MutualTest2\\
MutualTest1&:= exit1!.MutalTest + enter2!.bad!.0\\
MutualTest2&:= exit2!.MutalTest + enter2!.bad!.0
\end{align*}

Executando 
$$(Peterson|MutualTest)\backslash \{enter1,enter2,exit1,exit2\}$$
ver se a ação $bad!$ pode ser executada. Pode-se mostrar que não.
 Ficheiro no Pseuco.com: 
\url{tex}

\end{frame}



\end{document}
\documentclass[aspectratio=169]{beamer}

\usepackage{etex} % fixes new-dimension error
\usepackage{lmodern}
\usepackage[T1]{fontenc}

\usepackage{graphicx,amsmath}
\usepackage{stmaryrd} % cf. interleave
\usepackage{booktabs}
\usepackage{amscd}
\usepackage{multicol}
\usepackage[absolute,overlay]{textpos}
\usepackage{alltt}
\usepackage{proof}
\usepackage{subcaption}
\usepackage{tikz}
\usepackage{tikz-cd}
\usepackage[new]{old-arrows}
\usepackage[all]{xy}
\usepackage{pgfplots}
\usepackage{textcomp}

\usepackage{algorithmicx}
\usepackage[noend]{algpseudocode}
\usepackage[Algoritmo]{algorithm}

\input{src/macros/macros}

%-------------- template --------------------------------------------------
\input{src/macros/beamerconf}
%----------------------------------------------------------------------------
% \usepackage[utf8]{inputenc}
% %\usepackage[portuges]{babel}
% \usepackage{alltt}
% \usepackage{xspace}
% \usepackage{longtable}
% \usepackage{amsfonts}
% \usepackage{latexsym}
% \usepackage{amsmath}
% \usepackage{amssymb}
% %\usepackage{xcolor}
% \usepackage{enumerate}
% \usepackage{graphicx}
% %\usepackage{moreverb}
% %\usepackage{ifthen}
% \usepackage{url}
% \usepackage{tikz} 

% %\usetikzlibrary{snakes}
% \usetikzlibrary{decorations}
% \usetikzlibrary{shapes,shapes.geometric,arrows,fit,calc,positioning,automata,decorations.pathmorphing}
% \tikzset{state/.style={draw,ellipse}}

% \usepackage{algorithmicx}
% \usepackage[noend]{algpseudocode}
% \usepackage[Algoritmo]{algorithm}

\newcommand{\Versao}{pdf}

\newcommand{\lixo}[1]{}
% VFS6

\newcommand{\REs}{REs\xspace}
\newcommand{\Regfin}{\mathsf{Rfin}\xspace}
\newcommand{\dfa}{DFA\xspace}
\newcommand{\dfas}{DFAs\xspace}
\newcommand{\nfa}{NFA\xspace}
\newcommand{\enfa}{$\varepsilon$-NFA\xspace}
\newcommand{\enfas}{$\varepsilon$-NFAs\xspace}
\newcommand{\nfas}{NFAs\xspace}
\newcommand{\Set}[1]{\{\,#1\,\}}
\newcommand{\fun}{\rightarrow}
 \newcommand{\Implica}{\mbox{$\Rightarrow$}}
 \newcommand{\equival}{\mbox{$\Leftrightarrow$}}
 \newcommand{\halmos}{\mbox{$\Box$} \vspace{0.25in}}
 % \newcommand{\ie}{\mbox{\em i.e.}}
 \newcommand{\prim}{$1^{\mbox{\small \b{a}}}$\ }
 \newcommand{\eira}[1]{$#1^{\mbox{\small \b{a}}}$\ }
 \newcommand{\eiro}[1]{$#1^{\mbox{\small \b{o}}}$\ }
 \newcommand{\num}{$\mbox{n}^{\mbox{\small \b{o}}}$\ }
 % \newcommand{\set}[1]{\{#1\}}
 \newcommand{\Union}{\cup}
 \newcommand{\Inter}{\cap}
 \newcommand{\AND}{\;\land\;}
 \newcommand{\ANDe}{\;\land}
 \newcommand{\Or}{\;\lor\;}
 \newcommand{\Ore}{\;\lor}
 \newcommand{\Imp}{\to}
 \newcommand{\Impv}{\;\to\;}
 \newcommand{\Equiv}{\leftrightarrow}
 \newcommand{\True}{\mathsf{true}}
 \newcommand{\False}{\mathsf{false}}
 \newcommand{\NAND}{\tilde{\AND}}
 \newcommand{\NOR}{\tilde{\Or}}
 \newcommand{\ORE}{\dot{\lor}}
 \newcommand{\bool}{{\cal BOOL}}
 \newcommand{\EEquiv}{\Longleftrightarrow}
 \newcommand{\V}{{\color{red} \textbf{V}}}
 \newcommand{\Fa}{{\color{red} \textbf{F}}}

\newcommand{\Ident}{\equiv}



%MC
\newcommand{\kleene}[1]{\mbox{${#1}^\star$}}
\newcommand{\nter}[1] {\mbox{$\langle #1 \rangle$}}
\newcommand{\deriva}{\Rightarrow}
\newcommand{\produ}{\to}
\newcommand{\blank}{\bullet}
\newcommand{\esqa}{\mbox{$\gets$}}
\newcommand{\dira}{\mbox{$\to$}}

%LC
\newcommand{\Varp}{\mbox{${\cal V}_{Prop}$}}
\newcommand{\suf}{\mbox{$\Rightarrow$}}
\newcommand{\nec}{\mbox{$\Leftarrow$}}
\newcommand{\esimo}[2]{$#1^{\mbox{\small \b{#2}}}$\ }


%SD
\newcommand{\la}{\lambda}
\newcommand{\lc}{$\lambda$-calculus\xspace}
\newcommand{\lt}{$\lambda$-termo\xspace}
\newcommand{\lts}{$\lambda$-termos\xspace}
\newcommand{\LC}{\Lambda(\CC)}
\newcommand{\CLC}{{\cal CL}(\CC)}
\newcommand{\CL}{{\cal CL}}
\newcommand{\lapp}{\backslash}
\newcommand{\Rbu}[2]{ #1\rightarrow_{1\beta} #2 } 
\newcommand{\Rb}[2]{ #1\rightarrow_{\beta} #2 } 
\newcommand{\Ra}[2]{ #1\rightarrow_{\alpha} #2 } 
\newcommand{\Cb}[2]{ #1\equiv_{\beta} #2 } 
\newcommand{\Reu}[2]{ #1\rightarrow_{1\eta} #2 } 
\newcommand{\Ret}[2]{ #1\rightarrow_{\eta} #2 } 
\newcommand{\Ce}[2]{ #1\equiv_{\eta} #2 } 
\newcommand{\Rbeu}[2]{ #1\rightarrow_{1\beta\eta} #2 } 
\newcommand{\Rbet}[2]{ #1\rightarrow_{\beta\eta} #2 } 
\newcommand{\Cbe}[2]{ #1\equiv_{\beta\eta} #2 } 

%FLP

\newcommand{\Rw}{ \mapsto_{w} } 
\newcommand{\Rws}{ \mapsto_{w}^\star } 


\newcommand{\nums}[1]{\lceil\ #1\rceil}
\newcommand{\cpo}{($D$,$\sqsubseteq$)\ }

\newcommand{\cp}[1]{(#1,$\sqsubseteq$)}
\newcommand{\nn}{\mbox{$\mathbb{N}$}}
\newcommand{\nb}{\mbox{$\,\mathbb{N}_\bot\,$}}
\newcommand{\sse}{\Longleftrightarrow}
\newcommand{\ordu}{\sqsubseteq}
\newcommand{\nttr}{\;\nrightarrow\;}
\newcommand{\ttr}{\;\rightsquigarrow\xspace}
\newcommand{\tr}{\longrightarrow\xspace}
\newcommand{\trs}{\;\longrightarrow^\star\xspace}
\newcommand{\trtau}{\;\Rightarrow\;}
\newcommand{\tc}{\;\hookrightarrow\;}
\newcommand{\trd}{\;\rightarrow^d_v\;}
\newcommand{\confd}[2]{\langle #1,#2 \rangle}
\newcommand{\son}[3]{\confd{#1}{#2}\;\tr\;#3}
\newcommand{\sos}[4]{\confd{#1}{#2}\;\tr\;\confd{#3}{#4}}
\newcommand{\transt}[3]{#1\stackrel{#2}{\ttr}#3}
% \newcommand{\trans}[3]{#1\stackrel{#2}{\tr}#3}
\newcommand{\transs}[3]{#1\stackrel{#2}{\trs}#3}
\newcommand{\transc}[3]{#1\stackrel{#2}{\tc}#3}
\newcommand{\transi}[4]{#1\stackrel{#2}{\tr_{#4}}#3}
\newcommand{\transci}[4]{#1\stackrel{#2}{\tc_{#4}}#3}
\newcommand{\transtau}[3]{#1\stackrel{#2}{\trtau}#3}

\newcommand{\sta}[1]{\langle #1 \rangle}
\newcommand{\einf}{\stackrel{\infty}{\exists}}
\newcommand{\ainf}{\stackrel{\infty}{\forall}}
\newcommand{\sap}{(2^{AP})^\omega}
\newcommand{\sas}{(2^{AP})^\star}
\newcommand{\sam}{(2^{AP})^\plus}
\newcommand{\pw}{{\bf P}$\omega$}
\newcommand{\nat} {{\bf N}}
\newcommand{\fnn} {[\nat \longrightarrow \nat]_\bot}
\newcommand{\fff} {\fnn \longrightarrow \fnn}


\newcommand{\Lub}{\bigsqcup}


\newcommand{\rb}{\rightarrow_\beta}
\newcommand{\cb}{=_\beta}
\newcommand{\cbe}{=_{\beta\eta}}
\newcommand{\re}{\rightarrow_\eta}

%\newcommand{\vec}[1]{\overline{#1}}
\newcommand{\lsbra}{[\![}
\newcommand{\rsbra}{]\!]}
\newcommand{\bras}[1]{\lsbra  #1 \rsbra}
\newcommand{\brase}[1]{\lsbra  #1 \rsbra_\Gamma}
%\newcommand{\brass}[1]{Sat(#1)}
\newcommand{\sv}[2]{{\cal #1}\lsbra #2\rsbra}
\newcommand{\svs}[3]{{\cal #1}\lsbra #2\rsbra #3}
\newcommand{\svsc}[3]{{#1}\lsbra #2\rsbra #3}
\newcommand{\svsd}[4]{{#1}\lsbra #2\rsbra #3\, #4}
\newcommand{\svd}[2]{{#1}\lsbra #2\rsbra}
\newcommand{\svr}[3]{#1\lsbra #2\rsbra #3}
\newcommand{\svl}[2]{{\cal #1}\lsbra #2\rsbra\rho}
\newcommand{\svc}[2]{{\cal #1}\lsbra #2\rsbra\varphi\rho}
%\newcommand{\svc}[2]{{\cal #1}\lsbra #2\rsbra\rho\kappa}
%\newcommand{\svt}[2]{{\cal #1}\lsbra #2\rsbra\rho\theta}

\newcommand{\IFF}[3]{#1\,\fun \,#2,#3} 


\newcommand{\namesn}[1]{{#1}_{sn}}
\newcommand{\namess}[1]{{#1}_{ss}}
\newcommand{\namesd}[1]{{#1}_{sd}}

\newcommand{\fixp}[1] {\textsf{FIX}\; #1}

%VFS


\newenvironment{inducao}[2]{
  \begin{description}
      \setlength{\itemsep}{0cm}\setlength{\parsep}{0cm}
    \item[{\em Base.}] #1
    \item[{\em Indução.}] #2}
    {\end{description}}

\newenvironment{inducaoi}[1]{
    \begin{description}
      \setlength{\itemsep}{0cm}\setlength{\parsep}{0cm}
    \item[{\em Indução.}] #1}
    {\end{description}}

\newenvironment{inducaob}[1]{
    \begin{description}
      \setlength{\itemsep}{0cm}\setlength{\parsep}{0cm}
    \item[{\em Base.}] #1}
    {\end{description}}


\newenvironment{pequeno}{

\small

}{}

\newenvironment{mini}{

\tiny

}{}


 \newenvironment{twocol}[4]{

  \begin{tabular}{l|l}
    \begin{minipage}[t]{#1cm}
#2
    \end{minipage}
&
    \begin{minipage}[t]{#3cm}
#4
\end{minipage}
\end{tabular}
}{}






\newcounter{cont}
\newcounter{conti}

\newenvironment{dem}{{\bf Dem:} }

\newenvironment{proofu}{\setlength{\topsep}{-0.3cm}\setlength{\partopsep}{-0.3cm}\begin{description}\item[Dem.] }{\hfill $\Box$ \bigskip
\end{description}}

 \newenvironment{resolv}[1]{
 {\bf Resolução #1}

 }{}

 \newenvironment{coment}[1]{
 {\bf Comentário #1}

 }{}


\newenvironment{exerc}{\begin{exe} \it }{ 
$\diamond$ \end{exe}}

\newenvironment{exercicio}{\begin{Ex} \em }{ 
 \end{Ex}}

\newenvironment{resposta}[1]{{\parbox[t]{14cm}{{\bf R:}
  \vspace*{#1}}}}

\newenvironment{resposta1}[1]{{\parbox[t]{8cm}{{\bf R:} \vspace*{#1}}}}

\newcounter{taben}

\newenvironment{tabenum}{
\begin{enumerate}
  \setcounter{enumi}{\value{taben}}
  \renewcommand{\labelenumi}{\arabic{enumi}.}
  \renewcommand{\theenumi}{\arabic{enumi}.} }{\setcounter{taben}{\value{enumi}}\end{enumerate}}

%\usepackage{aula}

\newcounter{aula}

\newtheorem{exe}{Exercício}[aula]
\newtheorem{exem}{Exemplo}[aula]
\newtheorem{Ex}{Ex}[aula]
 \newtheorem{prob}{Problema}[aula]
 \newtheorem{lema}{Lema}[aula]
 \newtheorem{resol}{Resolução}[aula]
\newtheorem{teor}{Teorema}[aula]
\newtheorem{Def}{Definição}[aula]

\newtheorem{propos}{Proposição}[aula]
\newtheorem{cor}{Corolário}[aula]

%\newenvironment{saida}[1]{\fbox{\begin{verbatim} #1}{\end{verbatim}}}


\newcommand{\apply}{\textsc{apply}}
\newcommand{\restrict}{\textsc{restrict}}
\newcommand{\reduce}{\textsc{reduce}}
\newcommand{\foral}{\textsc{forall}}
\newcommand{\exist}{\textsc{exists}}


%\setlength{\itemsep}{0cm}

\tikzstyle{st}=[state, rectangle, rounded corners=7pt,  inner sep=4pt, minimum size=14pt]











% \newboolean{portuguese}
% \setboolean{portuguese}{true}
% %\setboolean{portuguese}{false}

% \newboolean{normal}
% \setboolean{normal}{true}

% \newcommand{\pten}[2]{\ifthenelse{\boolean{portuguese}}{#1}{#2}}
% \pten{\usepackage[portuges]{babel}}{\usepackage[british]{babel}}


\usepackage{bussproofs}

\begin{document}

% \setLectureBasic{Cyber-Physical Computation}
\setLecture[\\\emph{(slides mainly from Nelma Moreira)}]%
    {6}{CCS com passagem de valores}

	
\begin{frame}
\frametitle{Máquina de Café}

\only<1>{
O modelo básico
\begin{eqnarray*}
Machine&:=& coin?.coffee!.Machine	
\end{eqnarray*}

é muito simplista. Normalmente temos de  usar várias moedas
}
\only<2>{
\begin{eqnarray*}
Machine&:=& coin?.Machine+ coin?.PaidMachine\\
PaidMachine &:=& coffee?.(change!.Machine+\tau. 	Machine)
\end{eqnarray*}}
\only<3->{
Mas, suponham uma máquina em que o café custa 5 euros e aceita moedas de 1, 2 e 5 euros. Como tomar café? 
Solução: considerar todas as possibilidades... e troco
}
\only<4->{	\begin{eqnarray*}
	Machine0 &:=& coin1?.Machine1 + coin2?.Machine2 + coin5?.Machine5\\[-2pt]
Machine1 &:=& coin1?.Machine2 + coin2?.Machine3 + coin5?.Machine6\\[-2pt]
Machine2 &:=& coin1?.Machine3 + coin2?.Machine4 + coin5?.Machine7\\[-2pt]
Machine3 &:=& coin1?.Machine4 + coin2?.Machine5 + coin5?.Machine8\\[-2pt]
Machine4 &:=& coin1?.Machine5 + coin2?.Machine6 + coin5?.Machine9\\[-2pt]
Machine5 &:=& coffee?.Machine0\\[-2pt]
Machine6 &:=& coffee?.change1!.Machine0\\[-2pt]
Machine7 &:=& coffee?.change1!.change1!.Machine0\\[-2pt]
Machine8 &:=& coffee?.change1!.change1!.change1!.Machine0\\[-2pt]
Machine9 &:=& coffee?.change1!.change1!.change1!.change1!.Machine0
	\end{eqnarray*}}
	\only<5->{Por favor usar variáveis...}
\end{frame}	
	
\begin{frame}{Exemplo de um Protocolo  com erro no meio --Pseuco}
\only<1>{\begin{center}\includegraphics[width=7cm]{src/img/nam/SendRec1}\end{center}
}	
\only<2>{
\begin{eqnarray*}
Sender &:= &put? . send! . Sending\\[-2pt]
Sending& :=& receiveAck? . Sender +receiveNAck?.send!.Sending\\[-2pt]
Receiver &:= &receive? . get? . sendAck! . Receiver +\\[-2pt]
&&gargled?.sendNAck!.Receiver\\[-2pt]
	Medium& :=& send? . (receive! . Medium +i.garbled!.Medium)\\[-2pt]
AckMedium &:=& sendAck? . receiveAck! . AckMedium+\\[-2pt]
&&sendNAck?.receivedNAck!.AckMedium\\[-2pt]
DupMedium &:= &Medium | AckMedium\\[-2pt]
Protocol &:=& (Sender \mid Receiver \mid DupMedium) \backslash \\[-2pt]
&&\{send, receive, sendAck, receiveAck,\\[-2pt]
&&receiveNAck,sendNAck,garbled\}
\end{eqnarray*}}
\end{frame}

\begin{frame}{Envio de uma mensagem}
	\only<1>{
\begin{eqnarray*}
Sender &:= &put?x . send!x . Sending[x]\\[-2pt]
Sending[x]& :=& receiveAck? . Sender +receiveNAck?.send!x.Sending[x]\\[-2pt]
Receiver[x] &:= &receive?x . get?x . sendAck! . Receiver[x] +\\[-2pt]
&&gargled?.sendNAck!.Receiver[x]\\[-2pt]
	Medium& :=& send?x . (receive! x. Medium +i.garbled!.Medium)\\[-2pt]
AckMedium &:=& sendAck? . receiveAck! . AckMedium+\\[-2pt]
&&sendNAck?.receivedNAck!.AckMedium\\[-2pt]
DupMedium &:= &Medium | AckMedium\\[-2pt]
Protocol &:=& (Sender \mid Receiver \mid DupMedium) \backslash \\[-2pt]
&&\{send, receive, sendAck, receiveAck,\\[-2pt]
&&receiveNAck,sendNAck,garbled\}
\end{eqnarray*}
%$Protocol | put!1.put!2.put!3.put!4.0\backslash \{put\}$
}
\only<2>{
$$Protocol| put!2.put!4.put!2.put!8.0\backslash \{put\}$$
\begin{center}
\includegraphics[width=150pt]{src/img/nam/sendx.png}
\end{center}
}
\end{frame}

\begin{frame}{$CCS_{vp}$ com passagem de valor}
	\begin{itemize}[<+->]
		\item $a!v$: saída do valor $v$ no canal $a$ (enviar)
\item  $a?v$: entrada do valor $v$ pelo canal $a$ (receber)
\item ou usar variáveis 
	\item $a!x$: saída do valor guardado em $x$ no canal $a$ (enviar)
\item  $a?x$: entrada  de um valor que se guarda em $x$ pelo canal $a$ (receber)
\item e os nomes dos processos podem ter  variáveis com parâmetros permitindo assim enviar e receber valores ($A[x,y]$)
	\end{itemize}
\end{frame}

\begin{frame}
	\frametitle{Uma célula de memória (1-\emph{buffer})}
	\begin{eqnarray*}
		B&:=put?x.B1[x]\\
		B1[x]& := get!(x+1).B
	\end{eqnarray*}
	Então temos para
	$$(B|put!3.get?y.println!y)\{*,println\}$$
	o LTS
	
	\begin{center}
		\includegraphics[scale=.2]{src/img/nam/cell}
	\end{center}
\end{frame}

\begin{frame}{$CCS_{vp}$ com passagem de valor}
	Sendo $\mathbb{V}$ um conjunto de valores e $\mathbb{K}$ um conjunto de canais  temos 
\only<2->{
\begin{eqnarray*}
	A^!&=&\{a!v\mid a\in \mathbb{K},v\in \mathbb{V}\}\cup\{a!\mid a\in \mathbb{K}\}, \\
A^?&=&\{a?v\mid a\in \mathbb{K},v\in \mathbb{V}\}\cup\{a?\mid a\in \mathbb{K}\},\\
Com &=&  A^! \cup A^?\\
Act &=&Com \cup \{\tau\}
\end{eqnarray*}
}
\only<3->{
\begin{eqnarray*}
P&::=& 0 \;\mid\; X[r_1,\ldots ,r_n] \;\mid \;P+P\;  \mid \; \chi.P\;\mid\; P | P \;\mid\; P\backslash H \\
\chi& ::=& \tau \mid a! \mid a? \mid a!v \mid a?v \mid a!x \mid a?x  \end{eqnarray*}
onde $X\in Var$, $x\in D$, $r_i \in D\cup \mathbb{V} \cup \mathbb{K}$
}
\end{frame}

\begin{frame}{Regras do $CCS_{vp}$}
\only<1-4>{
\begin{prooftree}
	\LeftLabel{Pref} \AxiomC{ $\alpha \in Act $}
\UnaryInfC{$\trans{\alpha.P}{\alpha}{P}$}
\end{prooftree}
}
\only<2-3>{A avaliação de expressões $e\Downarrow z$: a expressão $e$ avalia para $z$.
}

\only<3>{\begin{prooftree}
	\LeftLabel{Output} \AxiomC{ $e \Downarrow z $}
\UnaryInfC{$\trans{a!e.P}{a!z}{P}$}
\end{prooftree}

\begin{prooftree}
	\LeftLabel{Valor} \AxiomC{$e \Downarrow z$}
\UnaryInfC{$\trans{a?e.P}{a?z}{P}$}
\end{prooftree}

\begin{prooftree}
	\LeftLabel{Input} \AxiomC{$v\in \mathbb{V}$}
\UnaryInfC{$\trans{a?x.P}{a?v}{P\{v/x\}}$}
\end{prooftree}
onde $P\{v/x\}$ é $P$ onde $x$ é substituito por $v$ (e $x$ não é uma ação)}
\only<4>{
\begin{prooftree}
	\LeftLabel{Input} \AxiomC{$v\in \mathbb{V}$}
\UnaryInfC{$\trans{a?x.P}{a?v}{P\{v/x\}}$}
\end{prooftree}
onde $P\{v/x\}$ é $P$ onde $x$ é substituito por $v$ (e $x$ não é uma ação)
\begin{eqnarray*}
	(a!y.P)\{v/x\}&=&a!y.P\{v/x\}\quad \text{ se } y\not=x\\[-2pt]
	(a!x.P)\{v/x\}&=&\alert{a!v.P\{v/x\}}\\[-2pt]
(a?y.P)\{v/x\}&=&a?y.P\{v/x\}\quad \text{ se }y\not=x\\[-2pt]
	(a?x.P)\{v/x\}&=&\alert{a?x.P}\\[-2pt]
X[x]\{v/x\}&=& \alert{X[v]}\\[-2pt]
X[y]\{v/x\}&=& X[y] \quad \text{ se }y\not=x
\end{eqnarray*}
Para as restantes expressões é passado para as subexpressões.
}
\only<article>
{\begin{exem}}

\only<5>{
\begin{eqnarray*}
	send!y.Sending[x]\{3/x,5/y\}&=& send!5.Sending[3]\\
send!y.Sending[x]\{3/x,5/y,receive/send\}&=&receive!5.Sending[3]
\end{eqnarray*}
}
\only<article>
{\begin{exem}}

	\only<5>{
\begin{eqnarray*}
	send!(x+y)\{3/x,5/y\}&=& sent !(8)\\
	3+5 &\Downarrow & 8
\end{eqnarray*}
	}
\only<article>
{\end{exem}
}


\only<article>
{\begin{exem}}
\only<6>{Calcular o LTS de 
$put?x:0..5.send!x.0$
\begin{center}
		\includegraphics[scale=.2]{src/img/nam/ps}
	\end{center}}
\only<article>
{\end{exem}}


\only<7->{
\begin{prooftree}
	\LeftLabel{Rec} \AxiomC{$\trans{P\{v_1/r_1,\ldots,v_n/r_n\}}{\alpha}{P'}$}
\AxiomC{$\Gamma(X[r_1,\ldots,r_n])=P$}
\BinaryInfC{$\trans{X[v_1,\ldots,v_n]}{\alpha}{P'}$}
\end{prooftree}
}
\only<article>
{\end{exem}}
\end{frame}


 
\begin{frame}
\frametitle{ $when$: Bloqueador Condicional }
Supomos $\mathbb{V}=\mathbb{Z}$ ($CCS_{vp}^{Z}$)
\only<article>
{\begin{exem}}
\only<1>{
\begin{eqnarray*}
	B[x]&:= &when(x<4) put?.B[x+1]+ when(x>0) get?.B[x-1]
\end{eqnarray*}
	$B[0]$ ou $B[5]$ o que fazem?
	}
\only<article>
{\end{exem}}

\only<2>{ 
A expressão
$when(b)P$ se $b$ é verdade comporta-se como $P$ senão bloqueia.

Vamos só considerar expressões  com inteiros.

\begin{prooftree}
	\LeftLabel{cond} \AxiomC{$\trans{P}{\alpha}{P'}$}
\AxiomC{$b\Downarrow True$}
\BinaryInfC{$\trans{when(b)P}{\alpha}{P'}$}
\end{prooftree}
}
\end{frame}

\begin{frame}
Acrescentamos à gramática
\begin{eqnarray*}
P&::=& 0 \;\mid\; X[r_1,\ldots ,r_n] \;\mid \;P+P\;  \mid \;\chi.P\;\mid\; P | P \;\mid\; P\backslash H\mid\alert{ when(b) P}\\
\chi& ::=& \tau \mid a! \mid a? \mid a!v \mid a?v \mid a!x \mid a?x  
\end{eqnarray*}

\end{frame}


\begin{frame}
\only<article>
{\begin{exem}}
\begin{eqnarray*}
Machine[b]&:=&when(b<5)coin?c.Machine[b+c]\\
&& +when(b\geq 5)coffee!.ReturnMachine[b-5]\\
ReturnMachine[b] &:=& when(b>0)change!.ReturnMachine[b-1]\\&&+Machine[0]\\
 User&:=&coin!2.coin!2.coin!2.coffee?.change?.0
\end{eqnarray*}

Calcula $\bras{(Machine[0] | User)\backslash\{coin,change,coffee\}}_\Gamma$.
\only<article>
{\end{exem}}
\end{frame}


\begin{frame}
\only<article>
{\begin{exem}}

\begin{eqnarray*}
	IterMult[z,x,y] &:=& when(x>0) i.IterMult[z+y,x-1,y] \\&&
        + when(x==0) println!z.0\\
IterMult[0,3,7]
\end{eqnarray*}

\only<article>
{\end{exem}}

\end{frame}

\begin{frame}{Regras  $CCS_{vp}^{Z}$}
\splittwo{0.6}{0.35}{
%\only<3->{
\begin{prooftree}
	\LeftLabel{Pref} \AxiomC{ $\alpha \in Act $}
\UnaryInfC{$\trans{\alpha.P}{\alpha}{P}$}
\end{prooftree}
%}
%\only<2->{
\begin{prooftree}
	\LeftLabel{Input} \AxiomC{$v\in \mathbb{V}$}
\UnaryInfC{$\trans{a?x.P}{a?v}{P\{v/x\}}$}	
\end{prooftree}

\begin{prooftree}
	\LeftLabel{Rec} \AxiomC{$\trans{P\{v_1/r_1,\ldots,v_n/r_n\}}{\alpha}{P'}$}
\AxiomC{$\Gamma(X[r_1,\ldots,r_n])=P$}
\BinaryInfC{$\trans{X[r_1,\ldots,r_n]}{\alpha}{P'}$}
\end{prooftree}
%}
}{
\begin{prooftree}
	\LeftLabel{Output} \AxiomC{ $e \Downarrow z $}
\UnaryInfC{$\trans{a!e.P}{a!.z}{P}$}
\end{prooftree}

\begin{prooftree}
	\LeftLabel{Valor} \AxiomC{$e \Downarrow z$}
\UnaryInfC{$\trans{a?e.P}{a?z}{P}$}
\end{prooftree}

%\only<3->{
\begin{prooftree}
	\LeftLabel{cond} \AxiomC{$\trans{P}{\alpha}{P'}$}
\AxiomC{$b\Downarrow True$}
\BinaryInfC{$\trans{when(b)P}{\alpha}{P'}$}
\end{prooftree}
%}
}
\end{frame}

\begin{frame}
\only<presentation>{\frametitle{Factorial}}
\only<article>
{\begin{exem} O exemplo do factorial}

\begin{eqnarray*}
Fac[n,j] &:= &when (j>0) i. Fac[n*j,j-1]\\
&&             + when (j==0) println!n. 0\\
\end{eqnarray*}
Calcular
$Fac[1,5]$

\only<article>
{\end{exem}	}
\end{frame}

\begin{frame}
\only<presentation>{\frametitle{Células de Memória (partilhada)}}
\only<article>{\begin{exem} Podemos definir um processo que corresponde a uma célula de memória que pode ser partilhada entre processos. Permite obter o valor guardado e guardar um valor}	

 
\only<1>{
\begin{eqnarray*}
Cell_x[cur]& :=& get_x!cur.Cell_x[cur] + set_x?new:0..2.Cell_x[new]
\end{eqnarray*}
}
\only<2->{ Mais geralmente:
	\begin{eqnarray*}
		Cell[rd,wr,x]&:=& rd!x.Cell[rd,wr,x]+wr?y.Cell[rd,wr,y]
	\end{eqnarray*}
}
\only<3->{
\begin{itemize}
	\item $Cell[rd,wr,5]$
\item $Cell[rdA,wrA,0] | Cell[rdB,wrB,0]$
\end{itemize}
}
\only<4->{
\begin{eqnarray*}
Cells&:=&Cell[rdA,wrA,0] | Cell[rdB,wrB,0]\\
Serve &:=& mult?. rdA?x:R. rdB?y:R. IterMult[0,x,y]\\
IterMult[z,x,y]& :=& when(x>0) i.IterMult[z+y,x-1,y] \\&&
        + when(x==0) println!z.Serve\\
Use &:=& wrA!7.wrB!5.mult!.0
\end{eqnarray*} 
$(Cells| Serve | Use )\backslash\{rdA,wrA,rdB,wrB,mult\}$

Qual o resultado?

}
\only<article>{\end{exem}}
\end{frame}

\begin{frame}\only<presentation>{
\frametitle{Factorial -  Versão iterativa}

\only<1->{\begin{eqnarray*}
Fac[n,j] &:= &when (j>0) i. Fac[n*j,j-1]\\
&&             + when (j==0) println!n. 0\\
\end{eqnarray*}
	}}
\only<article>
{\begin{exem} O exemplo do factorial em versao iterativa}
	
\only<2->{\begin{eqnarray*}
	Fak&:= &rdJ?j:R. ( when (j>0) rdN?n.wrN!(n*j).wrJ!(j-1).Fak\\
&&                + when (j==0) rdN?n.print!n.0 )\\
Cell[v,rd,wr]&:=& rd!v.Cell[v,rd,wr] + wr?x:R.Cell[x,rd,wr]\\
Cells& := &Cell[0,rdN,wrN] | Cell[0,rdJ,wrJ]\\
\end{eqnarray*}
$$(wrN!1.wrJ!5.Fak | Cells ) \backslash \{rdN,wrN,rdJ,wrJ\}$$
}

Ficheiro no Pseuco.com:

\url{https://pseuco.com/\#/edit/remote/5bglrm4937vze10dvlqi}
\only<article>
{\end{exem}}

\end{frame}

\begin{frame}{O $CCS_{vp}$ pode ser embebido no CCS}
..logo é só "syntatic sugar"...

\bigskip

\begin{center}
	
\begin{tabular}{|c|c|}\hline
$CCS_{vp}$& $CCS$ \\ \hline
	$a!v.P$
&$a_v!.P$\\\hline
$a?x.P$
&$\sum_{v\in \mathbb{V}}a_v?.P\{v/x\}$\\\hline
$X[u_1,\ldots,u_n]$&$X_{u_1,\ldots,u_n}$\\\hline
\end{tabular}
\end{center}

Isto é basta usar ações e nomes indexados, podendo ser considerados conjuntos infinitos de índices ($\mathbb{V}$ ou $D$)	
\end{frame}

\begin{frame}[fragile]
\frametitle{Controlo de Fluxo em $CCS_{vp}^\mathbb{Z}$}
\begin{columns}[T]
\begin{column}{0.47\textwidth}
O seguinte código 
{\small
\begin{verbatim}
while (a > 0) {
 println("loop");
 a = a-1;
}
println("a is zero");	
\end{verbatim}
}	
pode ser escrito em CCS
{\small
\begin{verbatim}
P[a] := when( a<=0)i.Q[a] 
        + when(a>0) println!"loop". P[a-1]
Q[a] :=println!"a is zero"
\end{verbatim}
}
\end{column}
\begin{column}{0.47\textwidth}
$P[3]$ teria o seguinte LTS 

\centering\includegraphics[width=170pt]{src/img/nam/loopc}
\end{column}
\end{columns}
\end{frame}

\end{document}